\documentclass{dissert}
\usepackage{cmap}
\usepackage[T2A]{fontenc}
\usepackage[utf8]{inputenc}

%\usepackage[cp1251]{inputenc}
\renewcommand{\rmdefault}{cmr}
\usepackage{dsfont} % for indicator function
\usepackage[russian]{babel}

\usepackage{color}
\usepackage{caption}
\usepackage{float}
%\usepackage{slashbox}
%пакеты и команды, необходимость в которых может возникнуть по ходу работы (Вы можете добавлять свои):
\usepackage{indentfirst}%для отступов
%\usepackage{subfig}

\usepackage{geometry}
\geometry{left=2.5cm}
\geometry{right=1.0cm}
\geometry{top=2.0cm}
\geometry{bottom=2.0cm}
\renewcommand{\baselinestretch}{1.0}

%\usepackage[backend=biber,style=alphabetic]{biblatex}    

%\usepackage{geometry}
%\emergencystretch=25pt%для борьбы с переполнениями за счет разреж. слов в абзаце
%\righthyphenmin=2% для разрешения переноса двух последних букв
%\arrayrulewidth=.75pt% регулируем толщину линий в табл.
%\usepackage[dvips]{graphicx}%для включения PS файлов
%\usepackage[final]{epsfig}
\usepackage{multicol}%для организации многоколоночного текста (предм. указатель)
\usepackage{subfigure}
\usepackage{indentfirst}
\usepackage{amsmath}
%\usepackage{enumerate}
\usepackage{amssymb}
\usepackage{amsthm}
\usepackage{amscd}
\usepackage{hhline}
%\usepackage{multirow}
\usepackage{graphicx, epsfig}
%\usepackage{epic}
\usepackage{amscd}
%\usepackage{ecltree}
\usepackage{color}
\usepackage[dvips,all]{xy}
%\usepackage{setspace}
%\usepackage{makeidx}
%пакет для вкл символов напр номер
%\usepackage{textcomp}
%\usepackage{amsfonts}
%\usepackage{afterpage}% Полезно для полного заполнения страниц перед большими таблицами
%\usepackage{longtable}%Для таблиц на нескольких страницах
%\usepackage{cite}
%\usepackage{rawfonts}
%\usepackage{oldlfont}%доступ к шрифтам через устаревшие команды
\usepackage{array}
%\renewcommand{\bibname}{Список литературы}
\renewcommand{\contentsname}{Содержание}
\renewcommand{\contentsdesc}{Стр.}
\renewcommand{\chaptername}{Глава}


%%% Библиография %%%
\makeatletter
\bibliographystyle{utf8gost71u}     % Оформляем библиографию по ГОСТ 7.1 (ГОСТ Р 7.0.11-2011, 5.6.7)
%\renewcommand{\@biblabel}[1]{#1.}   % Заменяем библиографию с квадратных скобок на точку
\makeatother

%\def\BibUrl#1.{}
%\bibliographystyle{gost2008} 
\theoremstyle{definition}
\newtheorem{theorem}{Теорема}
\newtheorem{lemm}{Лемма}
\newtheorem{defin}{Определение}
\newtheorem{utv}{Утверждение}

\def\bbljan{Январь}
\def\bblfeb{Февраль}
\def\bblmar{Март}
\def\bblapr{Апрель}
\def\bblmay{Май}
\def\bbljun{Июнь}
\def\bbljul{Июль}
\def\bblaug{Август}
\def\bblsep{Сентябрь}
\def\bbloct{Октябрь}
\def\bblnov{Ноябрь}
\def\bbldec{Декабрь}


\renewcommand{\thesubfigure}{\asbuk{subfigure}}

\graphicspath{{pic/}}


% opers
\DeclareMathOperator*{\indicator}{\mathds{1}}
\DeclareMathOperator*{\softmax}{softmax}
\DeclareMathOperator*{\idx}{idx}
\DeclareMathOperator*{\pos}{pos}
\DeclareMathOperator*{\AUCH}{AUCH}
\DeclareMathOperator*{\tf}{tf}
\DeclareMathOperator*{\ntf}{ntf}
\DeclareMathOperator*{\idf}{idf}
\DeclareMathOperator*{\ndf}{ndf}
\DeclareMathOperator*{\similarity}{sim}
\DeclareMathOperator*{\argmin}{arg\,min}
\DeclareMathOperator*{\const}{const}
\DeclareMathOperator*{\dBeta}{Beta}
\DeclareMathOperator*{\dDir}{Dir}
\DeclareMathOperator*{\Tr}{Tr}
\DeclareMathOperator*{\dDP}{DP}
\DeclareMathOperator*{\dMult}{Mult}
\DeclareMathOperator*{\dBern}{Bern}
\DeclareMathOperator*{\dCRP}{CRP}
\DeclareMathOperator*{\dKL}{KL}
\DeclareMathOperator*{\diag}{diag}
\newcommand{\dd}[1]{\mathrm{d}{#1}}

% BOLD
\newcommand{\bmatr}{{\mathbf{B}}}
\newcommand{\cmatr}{{\mathbf{C}}}
\newcommand{\hmatr}{{\mathbf{H}}}
\newcommand{\fmatr}{{\mathbf{F}}}
\newcommand{\mmatr}{{\mathbf{M}}}
\newcommand{\xmatr}{{\mathbf{X}}}
\newcommand{\pmatr}{{\mathbf{P}}}
\newcommand{\xmatrt}{{\tilde{\mathbf{X}}}}
\newcommand{\imatr}{{\mathbf{I}}}
\newcommand{\vmatr}{{\mathbf{V}}}
\newcommand{\wmatr}{{\mathbf{W}}}
\newcommand{\umatr}{{\mathbf{U}}}
\newcommand{\zmatr}{{\mathbf{Z}}}
\newcommand{\zmatrt}{{\tilde{\mathbf{Z}}}}
\newcommand{\Tmatr}{\mathbf{T}}
\newcommand{\lambdamatr}{{\mathbf{\Lambda}}}
\newcommand{\phimatr}{\mathbf{\Phi}}
\newcommand{\sigmamatr}{\mathbf{\Sigma}}
\newcommand{\thetamatr}{\boldsymbol{\Theta}}

\newcommand{\ab}{{\mathbf{a}}}
\newcommand{\bb}{{\mathbf{b}}}
\newcommand{\cb}{{\mathbf{c}}}
\newcommand{\db}{{\mathbf{d}}}
\newcommand{\eb}{{\mathbf{e}}}
\newcommand{\fb}{{\mathbf{f}}}
\newcommand{\gb}{{\mathbf{g}}}
\newcommand{\hb}{{\mathbf{h}}}
\newcommand{\mb}{{\mathbf{m}}}
\newcommand{\pb}{{\mathbf{p}}}
\newcommand{\qb}{{\mathbf{q}}}
\newcommand{\rb}{{\mathbf{r}}}
\newcommand{\tb}{{\mathbf{t}}}
\newcommand{\ub}{{\mathbf{u}}}
\newcommand{\vb}{{\mathbf{v}}}
\newcommand{\wb}{{\mathbf{w}}}
\newcommand{\xb}{{\mathbf{x}}}
\newcommand{\xt}{{\tilde{x}}}
\newcommand{\xbt}{\tilde{{\mathbf{x}}}}
\newcommand{\yb}{{\mathbf{y}}}
\newcommand{\zb}{{\mathbf{z}}}
\newcommand{\zt}{{\tilde{z}}}
\newcommand{\zbt}{{\tilde{\mathbf{z}}}}
\newcommand{\mub}{{\boldsymbol{\mu}}}
\newcommand{\alphab}{{\boldsymbol{\alpha}}}
\newcommand{\thetab}{\boldsymbol{\theta}}
\newcommand{\iotab}{\boldsymbol{\iota}}
\newcommand{\zetab}{\boldsymbol{\zeta}}
\newcommand{\xib}{\boldsymbol{\xi}}
\newcommand{\xibt}{\tilde{\boldsymbol{\xi}}}
\newcommand{\xit}{\tilde{\xi}}
\newcommand{\betab}{{\boldsymbol{\beta}}}
\newcommand{\phib}{{\boldsymbol{\phi}}}
\newcommand{\psib}{{\boldsymbol{\psi}}}
\newcommand{\gammab}{{\boldsymbol{\gamma}}}
\newcommand{\lambdab}{{\boldsymbol{\lambda}}}
\newcommand{\varepsilonb}{{\boldsymbol{\varepsilon}}}
\newcommand{\pib}{{\boldsymbol{\pi}}}


\newcommand{\scl}{s_{\mathsf{c}}}
\newcommand{\shi}{s_{\mathsf{h}}}
\newcommand{\shib}{\mathbf{s}_{\mathsf{h}}}
\newcommand{\MOD}{M}
\newcommand{\entr}{\mathsf{H}}
\newcommand{\REG}{\Omega}
\newcommand{\Mquol}{V}
%\newcommand{\prob}{\mathsf{P}}
\newcommand{\prob}{p}
\newcommand{\expec}{\mathsf{E}}

% overline
\newcommand{\xo}{{\overline{x}}}
\newcommand{\yo}{{\overline{y}}}

% bold overline
\newcommand{\xbo}{{\overline{\mathbf{x}}}}





\newcommand{\Amc}{{\mathcal{A}}}
\newcommand{\Bmc}{{\mathcal{B}}}
\newcommand{\Cmc}{{\mathcal{C}}}
\newcommand{\Jmc}{{\mathcal{J}}}
\newcommand{\Imc}{{\mathcal{I}}}
\newcommand{\Kmc}{{\mathcal{K}}}
\newcommand{\Lmc}{{\mathcal{L}}}
\newcommand{\Mmc}{{\mathcal{M}}}
\newcommand{\Nmc}{{\mathcal{N}}}
\newcommand{\Pmc}{{\mathcal{P}}}
\newcommand{\Tmc}{{\mathcal{T}}}
\newcommand{\Vmc}{{\mathcal{V}}}
\newcommand{\Wmc}{{\mathcal{W}}}


\newcommand{\T}{^{\text{\tiny\sffamily\upshape\mdseries T}}}
\newcommand{\deist}{\mathbb{R}}
\newcommand{\ebb}{\mathbb{E}}

%ALEX
\newcommand{\Amatr}{\mathbf{A}}
\newcommand{\X}{\mathbf{X}}
\newcommand{\Z}{\mathbf{Z}}
\newcommand{\Umatr}{\mathbf{U}}
\newcommand{\zetavec}{\boldsymbol{\zeta}}

\newcommand{\M}{\mathbf{M}}
\newcommand{\x}{{\mathbf{x}}}
\newcommand{\z}{{\mathbf{z}}}
\newcommand{\ical}{{\mathcal{I}}}
\newcommand{\tvec}{{\mathbf{t}}}
\newcommand{\xvec}{{\mathbf{x}}}
\newcommand{\zvec}{{\mathbf{z}}}
\newcommand{\bvec}{{\mathbf{b}}}
\newcommand{\qvec}{{\mathbf{z}}}
\newcommand{\pvec}{{\mathbf{p}}}
\newcommand{\wvec}{{\mathbf{w}}}
\newcommand{\rvec}{{\mathbf{r}}}
\newcommand{\thetavec}{{\mathbf{\theta}}}
\newcommand{\y}{{\mathbf{y}}}
\newcommand{\g}{{\mathbf{g}}}
\newcommand{\w}{{\mathbf{w}}}
\newcommand{\m}{{\mathbf{m}}}

%\renewcommand{\thesubfigure}{\asbuk{subfigure}}

\DeclareMathOperator*{\argmax}{arg\,max}



%different caption style: ``Fig.~N.~Caption text''
\makeatletter
\long\def\@makecaption#1#2{%
  \vskip\abovecaptionskip
  \sbox\@tempboxa{#1.~#2}%
  \ifdim \wd\@tempboxa >\hsize
    #1.~#2\par
  \else
    \global \@minipagefalse
    \hb@xt@\hsize{\hfil\box\@tempboxa\hfil}%
  \fi
  \vskip\belowcaptionskip}
\makeatother

\reversemarginpar

%\captionsetup[figure]{labelformat=gostfigure, justification=centering}
%\captionsetup[subfigure]{labelformat=gostfigure, justification=centering}
%\captionsetup[table]{labelformat=gostfigure, justification=centering}


\begin{document}


\thispagestyle{empty}


\begin{titlepage}
\begin{center}
\textsc{Московский физико-технический институт \\ (государственный университет)}\\
\end{center}
\vspace{1.5cm}
\begin{flushright}
{На правах рукописи\\
УДК 519.254}
\end{flushright}
\vspace{1.5cm}
\begin{center}
{Бахтеев Олег Юрьевич}
\par
\vspace{2cm}
\textsc{Последовательное порождение моделей \\глубокого обучения оптимальной сложности}
\par
\vspace{2cm}
{05.13.17~--- Теоретические основы информатики}
\par
\vspace{2cm}
{Диссертация на соискание ученой степени\\
кандидата физико-математических наук}
\end{center}
\vspace{2cm}
\hfill\parbox{8,4cm}{Научный руководитель:
\\д.ф.-м.н. В.\,В.\,Стрижов}
\par
\vspace{3.5cm}
\begin{center}
{Москва~--- 2018}
\end{center}
\end{titlepage}



\newpage{}

\tableofcontents{}%это оглавление
\newpage{}
\addcontentsline{toc}{section}{Введение}
\chapter*{Введение}


\textbf{Актуальность темы.} В работе рассматривается задача автоматического построения моделей глубокого обучения оптимальной и субоптимальной сложности. 

Под сложностью модели понимается \emph{минимальная длина описания}~\cite{mdl}, т.е. минимальное количество информации, которое требуется для передачи информации о модели и о выборке. Вычисление минимальной длины описания модели является вычислительно сложной процедурой. В работе предлагается получение ее приближенной оценки, основанной на связи минимальной длины описания и \emph{обоснованности модели}~\cite{mdl}. Для получения оценки обоснованности используются вариационные методы получения оценки обоснованности~\cite{bishop}, основанные на аппроксимации неизвестного апостериорного распределения другим заданным распределением. Под субоптимальной сложностью понимается вариационная оценка обоснованности модели.

Одна из проблем построения моделей глубокого обучения --- большое количество параметров моделей~\cite{hinton_rbm, hinton_init}. Поэтому задача выбора моделей глубокого обучения включает в себя выбор стратегии построения модели, эффективной по вычислительным ресурсам. В работе~\cite{greed} приводятся теоретические оценки построения нейросетей с использованием жадных стратегий,  при которых построение модели производится итеративно последовательным увеличением числа нейронов в сети. В работе~\cite{greed_mlp} предлагается жадная стратегия выбора модели нейросети с использованием релевантных априорных распределений, т.е. параметрических распределений, оптимизация параметров которых позволяет удалить часть параметров из модели. Данный метод был также применялся в задаче построения модели метода релевантных векторов~\cite{rvm}. Альтернативой данным алгоритмам построения моделей являются методы, основанные на прореживании сетей глубокого обучения~\cite{obd, popova, nvidia_prune}, т.е. последовательного удаления параметров, не дающих существенного прироста качества модели. 
В работах~\cite{Bengio, hd} рассматривается послойное построение модели с отдельным критерием оптимизации для каждого слоя. В работах~\cite{Kingma, gendis_pictures, gendis_phd} предлагается декомпозиция модели на порождающую и разделяющую, оптимизируемые последовательно. В работе~\cite{adanet} предлагается метод автоматического построения сети, основанный на бустинге. В качестве оптимизируемого функционала предлагается линейная комбинация функции правдоподобия выборки и сложности модели по Радемахеру. 
В работах~\cite{reinf,reinf_predict,reinf_deep2net,reinf_transfer} предлагается метод автоматического построения сверточной сети с использованием обучения с подкреплением. В~\cite{darts} используется схожее представление сверточной сети, вместо обучения с подкреплением используется градиентная оптимизация параметров, задающих структуру нейронной сети.

В качестве порождающих моделей в сетях глубокого обучения выступают ограниченные машины Больцмана~\cite{hinton_rbm} и автокодировщики~\cite{founds}. В работе~\cite{contractive} рассматриваются некоторые типы регуляризации автокодировщиков, позволяющие формально рассматривать данные модели как порождающие модели с использованием байесовского вывода. В работе~\cite{score} также рассматриваются регуляризованные автокодировщики и свойства оценок их правдоподобия. В работе~\cite{vae} предлагается обобщение автокодировщика с использованием вариационного байесовского вывода~\cite{bishop}. В работе~\cite{train_generative} рассматриваются модификации вариационного автокодировщика и ступенчатых сетей (англ. ladder network)~\cite{ladder} для случая построения многослойных порождающих моделей. 

В качестве критерия выбора модели в ряде работ~\cite{mackay,bishop,tokmakova,zaitsev,strijov_webber, strijov_dsc} выступает обоснованность модели. В работах~\cite{tokmakova,zaitsev,strijov_webber, strijov_dsc} рассматривается проблема выбора модели и оценки гиперпараметров в задачах регрессии. Альтернативным критерием выбора модели является минимальная длина описания~\cite{mdl}, являющаяся показателем статистической сложности модели и заданной выборки. 
В работе~\cite{perekrestenko} рассматривается перечень критериев сложности моделей глубокого обучения и их взаимосвязь. В работе~\cite{vladis} в качестве критерия сложности модели выступает показатель нелинейности, характеризуемый степенью полинома Чебышева, аппроксимирующего функцию. В работе~\cite{need_prune} анализируется показатель избыточности параметров сети. Утверждается, что по небольшому набору параметров в глубокой сети с большим количеством избыточных параметров можно спрогнозировать значения остальных. В работе~\cite{rob} рассматривается показатель робастности моделей, а также его взаимосвязь с топологией выборки и классами функций, в частности рассматривается влияние функции ошибки и ее липшицевой константы на робастность моделей. Схожие идеи были рассмотрены в работе~\cite{intrig}, в которой исследуется устойчивость классификации модели под действием шума. 

Одним из методов получения приближенного значения обоснованности является вариационный метод получения нижней оценки интеграла~\cite{bishop}. В работе~\cite{hoffman} рассматривается стохастическая версия вариационного метода. В работе~\cite{nips} рассматривается алгоритм получения вариационной нижней оценки обоснованности  для оптимизации гиперпараметров моделей глубокого обучения. В работе~\cite{varmc} рассматривается получение вариационной нижней оценки интеграла с использованием модификации методов Монте-Карло. В работе~\cite{early} рассматривается стохастический градиентный спуск в качестве оператора, порождающего распределение, аппроксимирующее апостериорное распределение параметров модели. Схожий подход рассматривается в работе~\cite{sgd_cont}, где также рассматривается стохастический градиентный спуск в качестве оператора, порождающего апостериорное распределение параметров. В работе~\cite{langevin} предлагается модификация стохастического градиентного спуска, аппроксимирующая апостериорное распределение. 

Альтернативным методом выбора модели является выбор модели на основе скользящего контроля~\cite{cv_ms, tokmakova}. Проблемой такого подхода является возможная высокая вычислительная сложность~\cite{expensive, expensive2}. В работах~\cite{bias,bias2} рассматривается проблема смещения оценок качества модели при гиперпараметрах, получаемых с использованием $k$-fold метода скользящего контроля, при котором выборка делится на $k$ частей с обучением на $k-1$ части и валидацией результата на оставшейся части выборки. 

Задачей, связанной с проблемой выбора модели, является задача оптимизации гиперпараметров~\cite{mackay,bishop}. В работе~\cite{tokmakova} рассматривается оптимизация гиперпараметров с использованием метода скользящего контроля и методов оптимизации обоснованности моделей, отмечается низкая скорость сходимости гиперпараметров при использовании метода скользящего контроля. В ряде работ~\cite{hyper, hyper2} рассматриваются градиентные методы оптимизации гиперпараметров, позволяющие оптимизировать большое количество гиперпараметров одновременно. В работе~\cite{hyper} предлагается метод оптимизации гиперпараметров с использованием градиентного спуска с моментом, в качестве оптимизируемого функционала рассматривается ошибка на валидационной части выборки. В работе~\cite{approx_hyper} предлагается метод аппроксимации градиента функции потерь по гиперпараметрам, позволяющий использовать градиентные методы в задаче оптимизации гиперпараметров на больших выборках. В работе~\cite{greed_hyper} предлагается упрощенный метод оптимизации гиперпараметров с градиентным спуском: вместо всей истории обновлений параметров для оптимизации используется только последнее обновление. В работе~\cite{sgd_cont} рассматривается задача оптимизации параметров градиентного спуска с использованием нижней вариационной оценки обоснованности. 


\vspace{0.5cm}
\textbf{Цели работы.}
\vspace{0.2cm}
\begin{enumerate}
\item Исследовать методы построения моделей глубокого обучения оптимальной и оптимальной сложности.
%\item Проанализировать различные подходы к решению задачи автоматического построения моделей глубокого обучения и оптимизации параметров модели.
\item Предложить критерии оптимальной и субоптимальной сложности модели глубокого обучения.
\item Предложить метод выбора субоптимальной структуры модели глубокого обучения.
\item Предложить алгоритм построения модели субоптимальной сложности и оптимизации параметров.
%\item Предложить алгоритм построения модели субоптимальной сложности и оптимизации параметров модели и
\end{enumerate}


\vspace{0.5cm}
\textbf{Методы исследования.} Для достижения поставленных целей используются методы вариационного байесовского вывода~\cite{mackay, bishop, early}. Рассматривается графовое представление нейронной сети~\cite{reinf,darts}. Для получения вариационных оценок обоснованности модели используется метод, основанный на градиентном спуске~\cite{sgd_cont, early}. В качестве метода получения модели субоптимальной сложности используется метод автоматического определения релевантности параметров~\cite{mackay,vae_ard} с использованием градиентных методов оптимизации гиперпараметров~\cite{hyper, hyper2, greed_hyper, approx_hyper}.

\vspace{0.5cm}
\textbf{Основные положения, выносимые на защиту.}
\vspace{0.3cm}
\begin{enumerate}
\item Предложен метод байесовского выбора субоптимальной структуры модели
глубокого обучения с использованием автоматического определения
релевантности параметров.
\item Предложены критерии оптимальной и субоптимальной сложности модели глубокого обучения.
\item Предложен метод графового описания моделей глубокого обучения.
%\item Проведено исследование свойства оптимизационных алгоритмов выбора модели.
\item Предложена задача оптимизации модели, обобщающая ранее
описанные методы выбора модели:
оптимизация обоснованности модели,
последовательное увеличение сложности модели,
последовательное снижение сложности модели,
полный перебор вариантов структуры модели.
\item Предложен метод оптимизации вариационной оценки обоснованности на основе мультистарта оптимизации модели.
\item Предложен алгоритм оптимизации параметров, гиперпараметров и структурных параметров моделей глубокого обучения.
\item Проведено исследование свойств оптимизационной задачи при различных значениях метапараметров. Рассмотрены ее асимптотические свойства.
\end{enumerate}


\vspace{0.5cm}
\textbf{Научная новизна.} Разработан новый подход к построению моделей глубокого обучения. Предложены критерии субоптимальной и оптимальной сложности модели, а также исследована их связь. Предложен метод построения модели глубокого обучения субоптимальной сложности. Исследованы методы оптимизации гиперпараметров и параметров модели.  Предложена обобщенная задача выбора модели глубокого обучения.

\vspace{0.5cm}
\textbf{Теоретическая значимость.} В целом, данная диссертационная работа носит теоретический характер. В работе предлагаются критерии субоптимальной и оптимальной сложности, основанные на принципе минимальной длины описания. Исследуется взаимосвязь критериев оптимальной и субоптимальной сложности. Предлагаются градиентные методы для получения оценок сложности модели. Доказывается теорема об оценке энтропии эмпирического распределения параметров модели, полученных под действием оператора оптимизации.
Доказывается теорема об обобщенной задаче выбора модели глубокого обучения.


\vspace{0.5cm}
\textbf{Практическая значимость.} Предложенные в работе методы предназначены для построения моделей глубокого обучения в прикладных задачах регрессии и классификации; оптимизации гиперпараметров полученной модели; выбора модели из конечного множества заданных моделей; получения оценок переобучения модели.


\vspace{0.5cm}
\textbf{Степень достоверности и апробация работы.} Достоверность результатов подтверждена математическими доказательствами, экспериментальной проверкой полученных методов на реальных задачах выбора моделей глубокого обучения; публикациями результатов исследования в рецензируемых научных изданиях, в том числе рекомендованных ВАК. Результаты работы докладывались и обсуждались на следующих научных конференциях.
\begin{enumerate}
\item ``Восстановление панельной матрицы и ранжирующей модели в разнородных шкалах'', Всероссийская конференция <<57-я научная конференция МФТИ>>, 2014.
\item ``A monolingual approach to detection of text reuse in Russian-English collection'', Международная конференция <<Artificial Intelligence and Natural Language Conference>>, 2015~\cite{monolingual}.
\item ``Выбор модели глубокого обучения субоптимальной сложности с использованием вариационной оценки правдоподобия'', Международная конференция <<Интеллектуализация обработки информации>>, 2016~\cite{ioi16}.
\item ``Machine-Translated Text Detection in a Collection of Russian
Scientific Papers'', Международная конференция по компьютерной лингвистике и интеллектуальным технологиям <<Диалог-21>>, 2017~\cite{dialog}.
\item ``Author Masking using Sequence-to-Sequence Models'', Международная конференция <<Conference and Labs of the Evaluation Forum>>, 2017~\cite{pan_s2s}.
\item ``Градиентные методы оптимизации гиперпараметров моделей глубокого обучения'', Всероссийская конференция <<Математические методы распознавания образов ММРО>>, 2017~\cite{mmro17_hyper}.
\item ``Детектирование переводных заимствований в текстах научных статей из журналов, входящих в РИНЦ'', Всероссийская конференция <<Математические методы распознавания образов ММРО>>, 2017~\cite{mmro17_plag}.
\item ``ParaPlagDet: The system of paraphrased plagiarism detection'', Международная конференция <<Big Scholar at conference on knowledge discovery and data mining>>, 2018.
\item ``Байесовский выбор наиболее правдоподобной структуры модели глубокого обучения'', Международная конференция <<Интеллектуализация обработки информации>>, 2018~\cite{ioi18}.
\item ``Variational learning across domains with triplet
information'', Международная конференция <<Visually Grounded Interaction and Language workshop, Conference on Neural Information Processing Systems>>, 2018.
\end{enumerate}

Работа поддержана грантами Российского фонда фундаментальных исследований.
\begin{enumerate}
\item 19-07-00875, Развитие методов автоматического построения и выбора вероятностных моделей субоптимальной сложности в задачах глубокого обучения.
\item 16-37-00488, Разработка алгоритмов построения сетей глубокого обучения как суперпозиций универсальных моделей.
\item 16-07-01158, Развитие теории построения суперпозиций универсальных моделей классификации сигналов.
\item 14-07-3104,  Построение и анализ моделей классификации для выборок малой мощности.
\end{enumerate}

\vspace{0.5cm}
\textbf{Публикации по теме диссертации.} Основные результаты по теме диссертации изложены в 11 печатных изданиях, 9 из которых изданы в журналах, рекомендованных ВАК.
\begin{enumerate}
\item О. Ю. Бахтеев, М. С. Попова, В. В. Стрижов, “Системы и средства глубокого обучения в задачах классификации”, Системы и средства информатики, 26:2 (2016), 4–22~\cite{popova2}.
\item Bakhteev, O., Kuznetsova, R., Romanov, A. and Khritankov, A., 2015, November. A monolingual approach to detection of text reuse in Russian-English collection. In 2015 Artificial Intelligence and Natural Language and Information Extraction, Social Media and Web Search FRUCT Conference (AINL-ISMW FRUCT) (pp. 3-10). IEEE~\cite{monolingual}.
\item Romanov, A., Kuznetsova, R., Bakhteev, O. and Khritankov, A., 2016. Machine-Translated Text Detection in a Collection of Russian Scientific Papers. Компьютерная лингвистика и интеллектуальные технолгии. 2016~\cite{dialog}. 
\item Bakhteev, O. and Khazov, A., 2017. Author Masking using Sequence-to-Sequence Models. In CLEF (Working Notes). 2017~\cite{pan_s2s}.
\item О. Ю. Бахтеев, В. В. Стрижов, “Выбор моделей глубокого обучения субоптимальной сложности”, Автоматика и телемеханика, 2018, № 8, 129–147; Automation Remote Control, 79:8 (2018), 1474–1488~\cite{var_ait}.
\item А. В. Огальцов, О. Ю. Бахтеев, “Автоматическое извлечение метаданных из научных PDF-документов”, Информатика и её применения, 12:2 (2018), 75–82~\cite{ogaltsov}.
\item А. Н. Смердов, О. Ю. Бахтеев, В. В. Стрижов, “Выбор оптимальной модели рекуррентной сети в задачах поиска парафраза”, Информатика и её применения, 12:4 (2018), 63–69~\cite{smerdov}.
\item Грабовой А.В., Бахтеев О.Ю., Стрижов В.В. “Определение релевантности параметров нейросети”, Информатика и её применения. 13:2 (2019), 62-71.
\item Bakhteev, O.Y. and Strijov, V.V., 2019. Comprehensive analysis of gradient-based hyperparameter optimization algorithms. Annals of Operations Research, pp.1-15~\cite{hyper_bakhteev}.


\item Бахтеев О.Ю. Восстановление панельной матрицы и ранжирующей модели по метризованной выборке в разнородных данных. // Машинное обучение и анализ данных. 2016. № 7. С. 72-77~\cite{panel}.
\item Бахтеев О.Ю. Восстановление пропущенных значений в разнородных шкалах с большим числом пропусков. // Машинное обучение и анализ данных. 2015. № 11. С. 1-11~\cite{knn}.

\end{enumerate}


\vspace{0.5cm}
\textbf{Личный вклад.} Все приведенные результаты, кроме отдельно оговоренных случаев, получены диссертантом лично при научном руководстве д.ф.-м.н. В. В. Стрижова.


\vspace{0.5cm}
\textbf{Структура и объем работы.} Диссертация состоит из оглавления, введения, четырех разделов, заключения, списка иллюстраций, списка таблиц, перечня основных обозначений и списка литературы из \total{citnum} наименований. Основной текст занимает \pageref{LastPage} страницы.

\vspace{0.5cm}
\textbf{Краткое содержание работы по главам.} В первой главе вводятся основные понятия и определения, формулируются задачи построения моделей глубокого обучения. Рассматриваются основные критерии выбора моделей. Рассматриваются существующие алгоритмы построения моделей глубокого обучения.

Во второй главе предлагается алгоритм построения субоптимальной модели глубокого обучения. Предлагаются методы оценки сложности модели.

В третьей главе исследуются методы оптимизации гиперпараметров модели.

В четвертой главе рассматривается задача выбора оптимальной и субоптимальной структуры модели глубокого обучения. Предлагается обобщающая задача выбора структуры модели глубокого обучения, исследуются ее асимптотические свойства. 

В пятой главе на базе предложенных методов описывается разработанный программный комплекс, позволяющий автоматически построить модель глубокого обучения субпотимальной сложности для заданной выборки для задачи классификации и регрессии. Работа данного комплекса анализируется на ряде выборок для задач классификации и регрессии. Результаты, полученные с помощью предложенных методов, сравниваются с результатами известных алгоритмов.

%\vspace{0.5cm}
%\textbf{Благодарности.}\\

\chapter{Постановка задачи}
\newpage{}
\addcontentsline{toc}{section}{Постановка задачи}
\chapter*{Постановка задачи}
Обработка тестовой информации является одной из наиболее важных задач
в области интеллектуального анализа данных. Теоретические результаты в дан-
ной области находят непосредственное применение при решении прикладных
задач, в частности, задач ранжирования поисковых выдач по запросу, задач
информационного поиска, анализа текстов, построения тематических моделей
коллекции текстов и терминологических словарей.


\chapter{Обзор}
\newpage{}
Задача выбора структуры модели является одной из базовых в области интеллектуального анализа данных.
Проблему выбора структуры модели глубокого обучения можно сформулировать следующим образом: решается задача классификации или регрессии на заданной выборке $\mathfrak{D}$. Требуется выбрать структуру нейронной сети, доставляющей минимум ошибки на этой функции и максимум качества на некотором внешнем критерии.
 Под моделью глубокого обучения понимается суперпозиция дифференцируемых нелинейный функций. Под структурой модели понимается значения структурных параметров модели, т.е. параметров модели, характеризующий вид итоговой суперпозиции. 

Формализуем описанную выше задачу.
Задана выборка \begin{equation}\label{eq:dataset}\mathfrak{D} = \{(\mathbf{x}_i,y_i)\}, i = 1,\dots,m,\end{equation} состоящая из множества пар <<объект-метка>> $$\mathbf{x}_i \in \mathbf{X} \subset \mathbb{R}^n, \quad {y}_i \in \mathbf{y} \subset \mathbb{Y}.$$ Метка ${y}$  объекта $\mathbf{x}$ принадлежит либо множеству: ${y} \in \mathbb{Y} = \{1, \dots, Z\}$ в случае задачи классификации, где $Z$ --- число классов, либо некоторому подмножеству вещественных чисел ${y} \in \mathbb{Y}  \subseteq \mathbb{R}$ в случае задачи регрессии. Далее будем полагать, что пары объект $(\mathbf{x}, y)$ являются реализацией некоторой случайно величины и порождены независимо.


\begin{defin}
Моделью глубокого обучения $\mathbf{f}$ назовем дифференцируемую по параметрам функцию:
\[
    \mathbf{f}(\mathbf{x}, \mathbf{W}): \mathbb{R}^n \to \mathbb{Y},
\]
где $\mathbf{W}$ --- вектор параметров функции $\mathbf{f}$.
\end{defin}

TODO: дальше идет определение структуры. Здесь тоже надо?

Для каждой модели определена функция правдоподобия  $p(\mathbf{y}|\mathbf{X}, \mathbf{W})$.
 

Смежной задачей к задаче выбора структуры модели является задача корректного представления структуры сети или параметризация сети глубокого обучения. Одним из возможных представлений структуры модели является графовое представление, в котором в качестве ребер графа выступают нелинейные функции, а в качестве вершин графа --- представление выборки под действием соответствующих нелинейных функций. 
Данный подход к описанию модели является достаточно общим и коррелирует с походом, описанным в~\cite{vokov}, а также в библиотеках типа TensorFlow, Caffe, Teano, Torch, в которых модель рассматривается как граф, ребрами которого выступают математические операции, а вершинами --- результат их действия на выборку. 
 В то же время, существуют и другие способы представления модели. В то же время, в ряде работ, посвященных байесовской оптимизации~\cite{snoek_deep,rbf_surrogate,bo_gp}, модель рассматривается как ``черный ящик'', имеющий ограниченный набор операций типа ``произвести оптимизацию параметров'' и ``предсказать значение зависимой переменной по независимой переменной''.
Подход, описанный в данных работах, также коррелирует с  библиотеками машинного обучения, такими как Weka, RapidMiner или sklearn, в которых модель машинного обучения рассматривается как ``черный ящик''.

\begin{defin}
Пусть задан граф $V,E$. Пусть для каждого ребра $(i,j) \in E$ определен вектор базовых функций $\mathbf{g}^{i,j}.$ Граф $V, E$ называется семейством моделей $\mathfrak{F}$, если функция, задаваемая рекурсивно как 
\[
    f_j(\mathbf{x}) = \sum_{i \in \text{Adj}(v_j)} \langle \boldsymbol{\gamma}^{i,j}, \mathbf{g}^{i,j} \rangle \left(\mathbf{f}_k(\mathbf{x})\right), \quad \mathbf{f}_0(\mathbf{x}) = \mathbf{x}
\]
является моделью при любых значениях векторов $\boldsymbol{\gamma}^{j,k}$.
\end{defin}

\begin{defin}
Параметрами модели $\mathbf{f}$ из семейства моделей $\mathfrak{F}$  назовем конкатенацию векторов параметров моделей $\{_{j=0}^|V| \mathbf{f}_j\}, \mathbf{W} \in \mathbb{R}^d.$
\end{defin}

\begin{defin}
Структурой модели $\boldsymbol{\Gamma}$ назовем конкатенацию векторов $\boldsymbol{\gamma}^{j,k}$.
\end{defin}

Будем полагать, что для параметров модели $\mathbf{W}$ и структуры  $\boldsymbol{\Gamma}$ задано некоторое априорное распределение $p(\mathbf{W}, \boldsymbol{\Gamma})$.
\begin{defin}
Гиперпараметрами модели $\mathbf{h}\in \mathbb{R}^h$ назовем параметры распределения $p(\mathbf{W}, \boldsymbol{\Gamma})$.
\end{defin}

\begin{defin}
Аппроксимирующим распределением назовем некоторое параметрическое приближение $q(\boldsymbol{\theta})$ апостериорного распределения параметров и структуры $p(\mathbf{W}, \boldsymbol{\Gamma}|\mathbf{X}, \mathbf{y}, \mathbf{h}).$ 
\end{defin}

\begin{defin}
Оптимизируемыми параметрами модели $\boldsymbol{\theta} \in \mathbb{R}^u$ назовем параметры аппроксимирующего распределения $q$.
\end{defin}

\begin{defin}
Пусть задано аппроксимирующее распределения $q$.
Функцией потерь $L(\boldsymbol{\theta}, \mathbf{h})$ для модели $\mathbf{f}$ назовем дифференцируемую функцию, характеризующую качество модели на обучающей выборки при параметрах модели, получаемых из  распределения $q$.
\end{defin}

В качестве функции $L$ может выступать правдоподобие и апостериорная вероятность параметров модели на обучающей выборке.

\begin{defin}
Пусть задано аппроксимирующее распределения $q$ и функция потерь $L$. 
Функцией валидации $Q(\mathbf{h},\boldsymbol{\theta})$ для модели $\mathbf{f}$ назовем дифференцируемую функцию, характеризующую качество модели при векторе $\boldsymbol{\theta}$, заданном неявно.
\end{defin}


В общем случае задача выбора структуры модели и параметров модели ставится как двухуровневая задача оптимизации:
\begin{equation}
\label{eq:optim}
	\mathbf{h}^{*} = \argmax_{\mathbf{h} \in \mathbb{R}^h} Q(\mathbf{h}),
\end{equation}
где $T$ --- оператор оптимизации, решающий задачу оптимизации:
\[
   \boldsymbol{\theta}^{*} = \argmax_{\boldsymbol{\theta} \in \mathbb{R}^u} L(\boldsymbol{\theta}, \mathbf{h}).
\]


Заметим, что частным случаем выбора структуры глубокой сети является выбор обобщенно-линейных моделей, т.к. отдельные слои нейросети можно рассматривать как обобщенно-линейные модели. Задачу выбора обобщенно-линейной модели можно рассматривать как задачу выбора признаков, методы решения которой делятся на три группы~\cite{feature_select}:
\begin{enumerate}
\item Фильтрационные методы. Основной особенностью данных методов является то, что такие методы не используют какой-либо информации о модели, а отсекают признаки только на основе статистических показателей. 
\item Оберточные методы --- методы, анализирующие подмножества признаков. Такие методы выбирают не признаки, а подмножества признаков, что позволяет учесть корреляция признаков.
\item Методы погружения проводят оптимизацию моделей и выбор признаков в единой процедуре, являясь комбинацией предыдущих типов отбора признаков.
\end{enumerate} 


\section{Метаоптимизация}
Задача выбора структуры модели тесно связана с раздел машинного обучения под названием \textit{метаобучение}. Под метаобучением понимаются алгоритмы машинного оубчения~\cite{metalearn}, которые:
\begin{enumerate}
\item могут оценивать и сравнивать методы оптимизации моделей
\item оценивать возможные декомпозиции процесса оптимизации моделей
\item на основе полученных оценок предлагать оптимальные стратегии оптимизации моделей и отвергать неоптимальные стратегии. 
\end{enumerate}



\subsection{Теоретические основы метаобучения }
В работе~\cite{layerwise_optimal} рассматривается задача построения порождающих моделей, предлагается критерий для послойного обучения порождающих моделей:
%\begin{figure}[H]
%\includegraphics[width=\textwidth]{./plots/arch_review_figs/mub.png}
%\end{figure}
$$
L = \max_{\boldsymbol{\theta}} ???
$$

\begin{defin}
Сэмплированием назовем порождение нескольких экземпляров модели из заданного аппроксимирующего распределения $q$.
\end{defin}

В работе~\cite{search_space} рассматриваются подходы к сэмплированию моделей глубокого обучения. Предлагается формализация пространства поиска и формальное описание элементов  пространства моделей:
 \begin{figure}[H]
\includegraphics[width=\textwidth]{./plots/arch_review_figs/search_space.png}
\end{figure}

\subsection{Метаоптимизация структуры моделей}
В работе~\cite{self_rnn} предлагается подход к адаптивному изменению структуры сети, основанный на обучении с подкреплением. Предлагается параметризация модели нейросети, включающая в себя модифицирующие и анализирующие выходы, позволяющие модифицировать параметры модели:
\begin{figure}[H]
\includegraphics[width=\textwidth]{./plots/arch_review_figs/self_rnn.png}
\end{figure}
Предлагается продолжение подхода, позволяющая рекуррентно продолжать анализ модели и порождать мета-мета-$\dots$-анализ.

В работе~\cite{meta_sgd} рассматривается оптимизация метапараметров (шага градиентного спуска и начального распределения параметров) с использованием обучения с подкреплением. На каждой итерации сэмплируется подвыборка, по которой проводится оптимизация данных метапараметров:
\begin{figure}[H]
\includegraphics[width=\textwidth]{./plots/arch_review_figs/meta_sgd.png}
\end{figure}

В работе~\cite{l2l} рассматирвается задача восстановления параметров модели по параметрам слабо обученной модели:
\begin{figure}[H]
\includegraphics[width=\textwidth]{./plots/arch_review_figs/l2l.png}
\end{figure}

\begin{figure}[H]
\includegraphics[width=\textwidth]{./plots/arch_review_figs/l2l_scheme.png}
\end{figure}

В работе~\cite{l2l_by_gd_gd} рассматривается оптимизация метапараметров оптимизации с помощью LSTM, которая выступает альтернативе аналитических алгоритмов, таких как Adam или AdaGrad. LSTM имеет небольшое количество параметров, т.к. для каждого метапараметра используется своя копия модели LSTM с одинаковыми параметрами для каждой копии:
\begin{figure}[H]
\includegraphics[width=\textwidth]{./plots/arch_review_figs/l2lbygd.png}
\end{figure}




\subsection{Случайное порождение структур}
В работе~\cite{search_smbo} рассматривается задача порождения сверточных нейронных сетей. Предлагается проводить поиск структуры сети по восходящему по сложности порядку: начиная от сетей с одни блоком и наращивая блоки. В силу высокой вычислительной сложности данного подхода, вместо построения модели,предлагается обучить рекуррентную нейросеть,которая предсказывает качество модели по заданным блокам. 


В работе~\cite{optimal_racing} рассматривается задача выбора архитектуры с помощью большого количества параллельных запусков обучения моделей, предлагаются критерии ранней остановки оптимизации обучения моделей.


\subsection{ Обучение с подкреплением в задаче выбора структур моделей}
В работе~\cite{reinf} представлена схема выбора архитектуры сверточной нейросети с использованием обучения с подкреплением. В качестве актора (контроллера) выступает рекуррентная нейронная сеть.
\begin{figure}[H]
\includegraphics[width=\textwidth]{./plots/arch_review_figs/reinf.png}
\end{figure}
В работе~\cite{reinf_predict} предлагается построение регрессионной модели для оценки финального качества модели и ранней остановки оптимизации моделей. Данный подход позволил существенно ускорить поиск моделей, представленный в работе~\cite{reinf}.
В работе~\cite{reinf_transfer} рассматривается задача переноса архитектуры нейросети, обученной на более простой выборки, на более сложную. Также предлагается параметризация пространства поиска, более делатьное, чем в~\cite{reinf}:
\begin{figure}[H]
\includegraphics[width=\textwidth]{./plots/arch_review_figs/reinf2.png}
\end{figure}

В отличие от предыдущих работ, в работе~\cite{reinf_deep2net} предлагается подход к инкрементальному обучению нейросети, основанном на модификации модели, полученной на предыдущем шаге. Рассматривается две операции над нейросетью:
\begin{itemize}
\item Расширение сети
\item Углубление сети
\end{itemize}

\begin{figure}[H]
\includegraphics[width=\textwidth]{./plots/arch_review_figs/deep2net.png}
\end{figure}



\section{Адаптивное изменение структуры}
В данном разделе собраны методы изменения структуры существующей модели. 

\textbf{Алгоритмы наращивания и прореживания параметров модели}
В работе~\cite{obd} предлагается удалять неинформативные параметры модели, где в качестве показателя информативности выступает следующий фнуционал: 
\begin{figure}[H]
\includegraphics[width=\textwidth]{./plots/arch_review_figs/obd.png}
\end{figure}
В работе~\cite{obs} было предложено развитие данного метода. В данной работе, в отличие от~\cite{obd} не вводится предположений о дигональности Гессиана функции ошибок, поэтому удаление неинформативных параметров модели производится точнее.

В работе~\cite{nips} был предложен метод, основанный на получении вариационной нижней оценки правдоподобия модели. В качестве критерия информативности параметра выступало отношение вероятности нахождения параметра в пределах априорного распределения к вероятности равенста параметра нулю:
\begin{figure}[H]
\includegraphics[width=\textwidth]{./plots/arch_review_figs/nips_var.png}
\end{figure}
Идея данного метода была развита в~\cite{bayes_compr}, где также используются вариационные методы. В отличие от предыдущей работы, в данной работе рассматривается ряд априорных распределений параметров, позволяющих прореживать модели более эффективно:
\begin{itemize}
\item Нормальное распределение с лог-равномерным распределением дисперсии, независимой для каждого нейрона:
\begin{figure}[H]
\includegraphics[width=\textwidth]{./plots/arch_review_figs/bayes_compr_group.png}
\end{figure}
\item Произвденеие двух половинных распределений Коши: одно ответственно за отдельный параметр, другое --- за общее распределение параметров:
\begin{figure}[H]
\includegraphics[width=\textwidth]{./plots/arch_review_figs/bayes_compr_cauchy.png}
\end{figure}
\end{itemize}

Смежной темой к прореижванию моделей выступает компрессия нейросетей. Основным отличием задачи прореживания и компрессии выступает эксплуатационное требование: если прореживание используется для получения оптимальной и наиболее устойчивой модели, то компрессия часто производится для сохранения памяти и основных эксплуатационных характеристик исходной модели (?).
В работе~\cite{nvidia_prune}
предлагается итеритавиное использование регуляризации типа DropOut~\cite{dropout} для прореживания модели. 
В работах~\cite{weight_quantization, weight_quantization2} используются методы снижения вычислительной точности представления парамеров модели на основе кластеризации весов.
В работе~\cite{weight_quantization2} предлагается метод компрессии, основанный на кластеризации значений параметров модели и представлении их в сжатом виде на основе кодов Хаффмана.

В работах~\cite{boost_res, adanet} предлагается наращивание моделей, основанное на бустинге. В работе рассматривается задача построения нейросетевых моделей специального типа:
\[
    \mathbf{f}_{t+1} = \sigma(\mathbf{f}_t) + \mathbf{f}_t,
\]
приводится параметризация модели, позволяющая рассматривать декомпозировать модель на слабые классификаторы.
В работе~\cite{adanet} на каждом шаге построения выбирается одно из двух расширений модели, каждое из которых рассматирвается как слабый классификатор:
1. Сделать модель шире
2. Сделать модель глубже
\begin{figure}[H]
\includegraphics[width=0.5\textwidth]{./plots/arch_review_figs/adanet.png}
\end{figure}
Построение модели заканчивается при условии снижении радемахереовской сложности:
\begin{figure}[H]
\includegraphics[width=0.5\textwidth]{./plots/arch_review_figs/rad.png}
\end{figure}



\section{Байесовские методы порождения и выбора моделей}
\subsection{Автоматическое определение релевантности параметров}
В работе~\cite{hyper} рассматривается задача оптимизации гиперпараметров.  Авторы предлагают оптимизировать константы $l_2$-регуляризации отдельно для каждого параметра модели, проводится параллель с методами автоматического определения релевантности параметров (ARD)~\cite{MacKay}.

В работе~\cite{ard} рассматривается метод ARD для снижения размерности скрытого пространства вариационных порождающих моделей: скрытая переменная параметризуется как  произведение некоторой случайной величины $\mathbf{z}$  на вектор, отвечающий за релевантность каждой компоненты скрытой переменной:
\begin{figure}[H]
\includegraphics[width=0.5\textwidth]{./plots/arch_review_figs/ard.png}
\end{figure}

\subsection{Суррогатный выбор моделей}
В работе~\cite{bo_gp} предлагается моделировать качество модели гауссовым процессом, параметрами которого выступают гиперпараметры исходной модели.
Модель, аппроксимирующая качество исходной модели, называется суррогатом. 

Одна из основных проблем использования гауссового процесса как суррогатной модели --- кубическая сложность оптимизации. В работе~\cite{random_gaus} предлагается использовать случайные подпространства гиперпараметров для ускоренной оптимизации.  В работе~\cite{gp_tree} предлагается кобминация из множества гауссовых моделей и линейной модели, позволяющая модели нелинейные зависимости гиперпараметров, а также существенно сократить сложность оптимизации. 

В работе~\cite{rbf_surrogate} предлагается рассматривать RBF-модель для аппроксимации качества исходной модели, что позволяет ускорить процесс оптимизации суррогатной модели. В~\cite{snoek_deep} рассматривается глубокая нейронная сеть в качестве суррогатной функции. Вместо интеграла правдоподобия, который оценивается в случае использования гауссового процесса в качестве суррогата, используется максимум апостериорной веротяности.

Важным параметром гауссовых процессов является функиия ядра гауссового процесса, полностью определяющая процесс в случае нулевого среднего. В работе~\cite{gp_fusion} предлагается функция ядра, определенная на графах:
    \[
    k(x,y) = r(d(x,y)),
    \]
где $d$ --- геодезическое расстояние между вершинами графа, $r$ --- некоторая вещественная функция (наверно положительно определенная, но это не указано явно в статье).
В работе~\cite{gp_arc} рассматривается задач выбора структуры нейросети, предлагается ядро специального вида, позволюящее учитывать только те гиперпапараметры, которые есть в обеих сравниваемых моделях: к примеру, для двуслойной и трехслойной нейросети будут уччитываться гиперпараметры, отвечающие только за первые два слоя. 
\begin{figure}[H]
\includegraphics[width=0.5\textwidth]{./plots/arch_review_figs/arc.png}
\end{figure}


\subsection{Адаптивное изменение структуры}

В работе~\cite{cib} рассматривается порождение unsupervised-моделей с использованием расширения процесса Индийского Буфета:
\begin{figure}[H]
\includegraphics[width=0.5\textwidth]{./plots/arch_review_figs/cib_eq.png}
\end{figure}
\begin{figure}[H]
\includegraphics[width=0.5\textwidth]{./plots/arch_review_figs/cib.png}
\end{figure}

В работе~\cite{cib_simple} предлагается упрощенная модель Индийского Буфета:
\begin{figure}[H]
\includegraphics[width=0.5\textwidth]{./plots/arch_review_figs/cib_simple.png}
\end{figure}

В работе~\cite{shirakawa2018dynamic} предлагается параметризация структуры модели с использованием Бернуллиевских величин:
каждая величина отвечает за включение или выключение слоя сети.



\subsection{Порождающие модели}
В работе~\cite{Kingma} было предложено обобщение вариационного автокодировщика на случай частичного обучения: 
итоговая модель вариационного автокодировщика является порождающей моделью, учитывающий метки объектов. 

В работе~\cite{vae_graph} рассматривается обобщение вариационного автокодировщика на случай более общих графических моделей. Рассматривается проводить оптимизацию сложных графических моделей в единой процедуре. Для вывода предлагается использовать нейронные сети.
Другая модификация вариационного автокодировщика представлена в работе~\cite{vae_stick}, авторы рассматривают использование процесса сломанной трости в вариационном автокодировщике, тем самым получая модель со стохастической размерностью скрытой переменной. В работе~\cite{vae_mix} рассматривается смесь автокодировщиков, где смесь моделируется процессом Дирихле.


В работе~\cite{var_boost} предлагается подход к оптимизации неизвестного распределения с помощью вариационного вывода. Авторы предлагают решать задачу оптимизации итеративно, добавляя в модель новые компоненты вариационного распределения, проводится аналогия с бустингом.
\subsection{Состязательные модели}

\section{Прогнозирование графовых структур моделей}
В разделе собраны ключевые работы по порождению графовых моделей.

В работе~\cite{jaakkola2010learning} предлагается метод прогнозирования графовой структуры на основе линейного программирования. Предлагается свести проблему поиска графовой структуры к комбинаторной проблеме.

В работе~\cite{double_rnn} предлагается метод прогнозирования структур деревьев, основанный на дважды-рекуррентных нейросетях (doubly-reccurent), т.е. на сетях, отдельно предсказывающих глубину и ширину уровней деревьев.
\begin{figure}[H]
\includegraphics[width=0.5\textwidth]{./plots/arch_review_figs/jaakkola.png}
\end{figure}


\section{Прикладные методы выбора моделей}
\subsection{Эвристические методы}
В работе~\cite{layer_probe} предлагается метод анализа структуры сети на основе линейных классификаторов, построенных на промежуточных слоях нейросети.
Схожий метод был предложен в~\cite{branches}, где классификаторы на промежуточных уровнях используются для уменьшения вычислений при выполнении вывода и предсказаний.
Промежуточные классификаторы.работают как решающий список
http://www.eecs.harvard.edu/~htk/publication/2016-icpr-teerapittayanon-mcdanel-kung.pdf

В работе~\cite{nn_inc} предлагается инкрементальный метод построения нейросети: на каждом этапе построения в модель добавляются новые слои. Для улучшения качества модели, слои добавляются в начало модели, и затем проходят оптимизацию.

\subsection{Структуры сетей специального вида}
В данном разделе представлены работы по поиску оптимальной структуры сети, описывающие частные случаи поиска оптимальных моделей со структурами специального вида.

В работе~\cite{mixed} рассматривается оптимизация моделей нейросетей с бинарной функцией активацией. Задача оптимизации сводится к задаче mixed integer программирования, которая решается методами выпуклого анализа.


SKIP-сети, нужно ли писать? ResNet?
%https://papers.nips.cc/paper/6205-swapout-learning-an-ensemble-of-deep-architectures.pdf
%https://arxiv.org/pdf/1603.09382.pdf

%https://arxiv.org/pdf/1711.03130.pdf
В работе~\cite{energynet} предлагается метод построения сети глубокого обучения, структура которой выбирается с использованием обучения без учителя. Критерий оптимальности модели использует оценки энергитических функций и ограниченной машины Больцмана.

%https://arxiv.org/pdf/1511.02954.pdf
%https://arxiv.org/pdf/1701.08734.pdf
В работах~\cite{pathnet, supernet} рассматривается выбор архитектуры сети с использованием \textit{суперсетей}: больших связанных между собой сетей, образующих граф, пути в котором определяют итоговую архитектуру нейросети. В работе~\cite{supernet} рассматриваются стохастические суперсет, позволяющие выбрать структуру нейросети за органиченное время оптимизации. 
Схожий подход был предложен в работе~\cite{pathnet}, где предлагается использовать эволюционные алгоритмы для запоминания оптимальных подмоделей и переноса этих моделей в другие задачи.
\begin{figure}[H]
\includegraphics[width=0.5\textwidth]{./plots/arch_review_figs/supernets.png}
\end{figure}

%https://arxiv.org/pdf/1511.02954.pdf 
%https://arxiv.org/pdf/1511.05641.pdf
%https://arxiv.org/pdf/1701.03281.pdf
В работах~\cite{net2net, morph, partition} рассматриваются методы деформации нейросетей. 
В работе~\cite{partition} предлагается метод оптимального разделения нейросети на несколько независимых сетей для уменьшения количества связей и, как следствие, уменьшения сложности оптимизации модели. В работе~\cite{net2net} предлагается метод сохранения результатов оптимизации нейросети при построении новой более глубокой или широкой нейросети. 
В работе~\cite{morph} рассматривается задача расширения сверточной нейросети, нейросеть рассматривается как граф.





\chapter{Выбор модели с использованием вариационного вывода}
В данной главе рассматривается задача выбора моделей глубокого обучения субоптимальной сложности. Под сложностью модели понимается правдоподобие модели~\eqref{eq:evidence}. Под субоптимальной сложностью понимается приближенная оценка правдоподобия модели, полученная с использованием вариационных методов. Вводятся вероятностные предположения о распределении параметров. На основе байесовского вывода предлагается функция правдоподобия модели. Для получения оценки правдоподобия применяются вариационные методы с использованием градиентных алгоритмов оптимизации. Проводится вычислительный эксперимент на нескольких выборках.

В данной работе предлагается метод получения вариационной нижней оценки  правдоподобия модели с использованием модифицированного алгоритма стохастического градиентного спуска. {Модификация заключается в добавлении шумовой компоненты. Эта компонента позволяет получить более точные оценки правдоподобия модели для сравнения моделей и выбора наиболее адекватной из них. } Рассматривается ряд модификаций базового алгоритма. {В качестве базового алгоритма выступает алгоритм оптимизации параметров модели с использованием стохастического градиентного спуска без контроля переобучения. Он заключается в итеративном вычислении градиента по параметрам от функции правдоподобия обучающей выборки и изменении значений параметров с его учетом.} Приводится сравнение с алгоритмом получения вариационной нижней оценки, представленном в~\cite{nips}. {Рассматриваются следующие модификации базового алгоритма:
%\begin{enumerate}
%\item 
оптимизация с кросс-валидацией с использованием и без использования регуляризации модели,
алгоритм получения вариационной оценки правдоподобия модели с применением нормального распределения,
алгоритм получения вариационной оценки правдоподобия с использованием стохастического градиентного спуска,
алгоритм получения вариационной оценки правдоподобия с использованием стохастической динамики Ланжевена.}
 { Данные алгоритмы решают следующие проблемы оптимизации моделей  градиентным спуском: оптимизация модели с меньшими затратами вычислительных ресурсов, быстрая сходимость оптимизации, контроль переобучения и выбор наиболее адекватной модели.
Под переобучением понимается потеря обобщающей способности модели с увеличением правдоподобия обучающей выборки~\cite{MacKay}. Переобучение характерно для моделей с большим количеством параметров, сопоставимым с мощностью обучающей выборки, что встречается в случае выбора моделей глубокого обучения~\cite{hinton_rbm, suts}.
}
Также алгоритмы имеют дальнейшую возможность применения к градиентным алгоритмам оптимизации гиперпараметров, описанным в~\cite{hyper}.

Свойства представленных в данной работе  алгоритмов исследуется на выборках, на которых проверялась работа алгоритма вероятностного обратного распространения ошибок~\cite{pbp}, где авторы акцентируются на оптимизации параметров модели. 


\section{Постановка задачи оптимизации правдоподобия моделей}
Определим понятие статистической сложности модели. Сложностью модели будем называть \textit{правдоподобие модели}~\eqref{eq:evidence}.
Пусть задано множество моделей $M$, для которых, возможно, не определена общая параметризация.
Для каждой модели $\mathbf{f} \in {M}$ заданы различные значения гиперпараметров $\mathbf{h}$. 
Рассмотрим два подхода к сравнению сложностей моделей:
\begin{enumerate}
\item Модели $\mathbf{f}$ принадлежат одному семейству $\mathfrak{F}$. При таком подходе сравнение сложности различных моделей является  адекватным, т.к. они определены на общем пространстве структур $\amsmathbb{\Gamma}$ и параметров $\mathbb{W}$. Недостатком такого подхода является сложность вычисления правдоподобия модели в случае, когда структура $\boldsymbol{\Gamma}$ определена однозначно, что может противоречить введенным вероятностным предположениям о структуре модели.
\item Модели $\mathbf{f}$  рассматриваются независимо от общей параметризации. Недостатком такого подхода является возможная некорректность сравнения моделей с заведомо различными структурами моделей, сильно отличающимися по количеству параметров. Возможным решение данного недостатка является введение дополнительного штрафа за большое количество параметров в модели~\cite{MacKay}.
\end{enumerate}
В данном разделе рассматривается второй вариант. Будем полгать, что структура модели $\boldsymbol{\Gamma}$ для вероятностной модели глубокого обучения $\mathbf{f}$ определена однозначно. Тем не менее, основная часть данной главы также применима и ко первому варианту.

\begin{defin} Сложностью модели $\mathbf{f}$ назовем правдоподобие модели:
\begin{equation}
\label{eq:model_evidence}
	p(\mathbf{y}|\mathbf{X},\mathbf{h}) = \int_{\mathbf{w} \in \mathbb{W}} p(\mathbf{y}|\mathbf{X},\mathbf{w}, \mathbf{h})p(\mathbf{w}|\mathbf{h})d\mathbf{w}.
\end{equation}
\end{defin}


\begin{defin}Модель классификации $\mathbf{f}$ назовем оптимальной среди моделей $M$, если достигается максимум интеграла~\eqref{eq:model_evidence}.
\end{defin}


Требуется найти оптимальную модель $\mathbf{f}$ из заданного множества моделей $M$, а также значения ее параметров $\mathbf{w}$, доставляющие максимум апостериорной вероятности
\begin{equation}
\label{eq:var_inf_posterior}
	p(\mathbf{w}|\mathbf{y},\mathbf{X},\mathbf{h}) = \frac{p(\mathbf{y}|\mathbf{X}, \mathbf{w}, \mathbf{h})p(\mathbf{w}|\mathbf{h})}{p(\mathbf{y}|\mathbf{X}, \mathbf{h})}.
\end{equation}


%\begin{example_empty}
\begin{example}
Рассмотрим задачу линейной регрессии:
\[
	\mathbf{y} =\mathbf{X} \mathbf{w} + \boldsymbol{\varepsilon},\quad \boldsymbol{\varepsilon}  \sim \mathcal{N}(\mathbf{0},\mathbf{1}),\quad \mathbf{w} \sim  \mathcal{N}(\mathbf{0},\mathbf{A}^{-1}),
\]
где $\mathbf{A}$ --- диагональная матрица. 
Правдоподобие зависимой переменной имеет вид
\begin{equation}
\label{eq:example1}
	p(\mathbf{y}|  \mathbf{X}, \mathbf{w}, \mathbf{h}) = (2\pi) ^{-\frac{m}{2}} \textnormal{exp} \bigl(-\frac{1}{2}(\mathbf{y} -\mathbf{X} \mathbf{w})^\mathsf{T}(\mathbf{y} - \mathbf{X}\mathbf{w})\bigr),
\end{equation}
априорное распределение параметров модели имеет вид
\begin{equation}
\label{eq:prior}	
p(\mathbf{w}|\mathbf{h}) =  (2\pi) ^{-\frac{n}{2}} |\mathbf{A}|^{\frac{1}{2}} \textnormal{exp} (-\frac{1}{2}\mathbf{w}^\mathsf{T}\mathbf{A}\mathbf{w}).
\end{equation}

Правдоподобие модели~\eqref{eq:evidence} в этом примере вычисляется аналитически~\cite{hyperopt}:
\begin{equation}
\label{eq:ground}
	p(\mathbf{y}|\mathbf{X},\mathbf{h})  =  (2\pi) ^{-\frac{m}{2}} |\mathbf{A}|^{\frac{1}{2}} |\mathbf{H}|^{-\frac{1}{2}}  \textnormal{exp}\bigl(-\frac{1}{2}(\mathbf{y} -\mathbf{X} \hat{\mathbf{w}})^\mathsf{T}(\mathbf{y} - \mathbf{X}\hat{\mathbf{w}})\bigr)\textnormal{exp} \bigl(-\frac{1}{2}\hat{\mathbf{w}}^\mathsf{T}\mathbf{A}\hat{\mathbf{w}}\bigr),
\end{equation}
где $\hat{\mathbf{w}}$ --- значение наиболее вероятных~\eqref{eq:posterior} параметров модели:
\[
	\hat{\mathbf{w}} = \argmax p(\mathbf{w}|\mathbf{y}, \mathbf{X}, \mathbf{h}) = (\mathbf{A} + \mathbf{X}^\mathsf{T}\mathbf{X})^{-1}\mathbf{X}^\mathsf{T}\mathbf{y},
\]
$\mathbf{H}$ --- гессиан функции потерь $L$ модели:
\[
	\mathbf{H}	= \nabla \nabla_\mathbf{w} \left(\frac{1}{2} (\mathbf{y} -\mathbf{X} {\mathbf{w}})^\mathsf{T}(\mathbf{y} - \mathbf{X}{\mathbf{w}}) + \frac{1}{2}\mathbf{w}^\mathsf{T}\mathbf{A}\mathbf{w} \right) = \mathbf{A} + \mathbf{X}^\mathsf{T}\mathbf{X},
\]

\[ 
	L = - \textnormal{log} p(\mathbf{y}|  \mathbf{X}, \mathbf{w}, \mathbf{h}). 
\]
\end{example}

\begin{example}
Рассмотрим задачу классификации, в которой модель --- нейросеть с softmax-слоем на выходе:
\[
\mathbf{f} = \mathbf{f}_\textnormal{|V|}(\mathbf{f}_{|V|-1}(\dots \mathbf{f}_1(\mathbf{x}))),
\]
$\mathbf{f}_1, \dots, \mathbf{f}_|V|$ --- дифференцируемые функции, $\mathbf{f}_\textnormal{|V|}$ --- многомерная логистическая функция:
\[
	\mathbf{f}_\textnormal{|V|} = \frac{\mathbf{f}_{|V|-1}(\dots \mathbf{f}_1(\mathbf{x}))}{\sum_{r=1}^Z \textnormal{exp}\bigl( {f}^{r}_{|V|-1}(\dots \mathbf{f}_1(\mathbf{x})) \bigr)},
\]
где ${f}_{|V|-1}^{r}$ --- $r$-я компонента функции $\mathbf{f}_{|V|-1}$. Компонента $r$ вектора $\mathbf{f}_{|V|}$ определяет вероятность принадлежности объекта $\mathbf{x}$ к классу $r$. Логарифм правдоподобия зависимой переменной аналогично~\eqref{eq:example1} имеет вид
\[
	\textnormal{log} p({y}|\mathbf{x}, \mathbf{w}, \mathbf{h}) =  \textnormal{log}~{f}^{y}_{|V|} (\mathbf{f}_{|V-1|}(\dots \mathbf{f}_1(\mathbf{x}))).
\]

Данная модель описывает многослойную сеть, аналогичную моделям семейства, представленного на Рис.~\ref{fig:scheme_mlp}.
\end{example}

Интеграл правдоподобия~\eqref{eq:model_evidence} модели является трудновычислимым для данного семейства моделей. Одним из методов вычисления приближенного значения правдоподобия является получение вариационной оценки правдоподобия.  


{В качестве функции, приближающей логарифм интеграла~\eqref{eq:model_evidence}, будем рассматривать его нижнюю оценку, полученную при помощи неравенства Йенсена~\cite{Bishop}. Получим нижнюю оценку логарифма правдоподобия модели, используя неравенство}
\begin{equation} 
\label{eq:var_elbo}
\textnormal{log}~p(\mathbf{y}|\mathbf{X},\mathbf{h})  = \int_{\mathbf{w}} q(\mathbf{w})\textnormal{log}~\frac{p(\mathbf{y},\mathbf{w}|\mathbf{X},\mathbf{h})}{q(\mathbf{w})}d\mathbf{w} + \textnormal{D}_\textnormal{KL}  \bigl(q(\mathbf{w})||p(\mathbf{w}|\mathbf{y}, \mathbf{X}, \mathbf{h})\bigr) \geq	
\end{equation} 
$$
\geq \int_{\mathbf{w}} q(\mathbf{w})\textnormal{log}~\frac{p(\mathbf{y},\mathbf{w}|\mathbf{X},\mathbf{h})}{q(\mathbf{w})}d\mathbf{w} =
$$

$$
= -\textnormal{D}_\textnormal{KL} \bigl(q(\mathbf{w})||p(\mathbf{w}|\mathbf{h})\bigr) + \int_{\mathbf{w}} q(\mathbf{w})\textnormal{log}~{p(\mathbf{y}|\mathbf{X},\mathbf{w},\mathbf{h})} d \mathbf{w},
$$
где $\textnormal{D}_\textnormal{KL}\bigl(q(\mathbf{w})||p(\mathbf{w} |\mathbf{h})\bigr)$ --- расстояние Кульбака--Лейблера между двумя распределениями: $$\textnormal{D}_\textnormal{KL}\bigl(q(\mathbf{w})||p(\mathbf{w} |\mathbf{h})\bigr) = -\int_{\mathbf{w}} q(\mathbf{w})\textnormal{log}~\frac{p(\mathbf{w} | \mathbf{h})}{q(\mathbf{w})}d\mathbf{w},$$
$$
p(\mathbf{y},\mathbf{w}|\mathbf{X},\mathbf{h}) = p(\mathbf{y}|\mathbf{X},\mathbf{h})p(\mathbf{w}|\mathbf{h}).
$$

{
\begin{defin} Вариационной оценкой логарифма правдоподобия модели~\eqref{eq:model_evidence} $\textnormal{log}~p(\mathbf{y}|\mathbf{X},\mathbf{h})$ называется оценка $\textnormal{log}~\hat{p}(\mathbf{y}|\mathbf{X},\mathbf{h})$, полученная аппроксимацией неизвестного апостериорного распределения $p(\mathbf{w}| \mathbf{y}, \mathbf{X}, \mathbf{h})$ заданным распределением $q(\mathbf{w})$.
\end{defin}
}


Будем рассматривать задачу нахождения вариационной оценки как задачу оптимизации. Пусть задано множество распределений $\mathfrak{Q} =\{q(\mathbf{w})\}$. Сведем задачу нахождения наиболее близкой вариационной нижней оценки интеграла~\eqref{eq:evidence} к оптимизации вида
\[
     \hat{q}(\mathbf{w}) = \argmax_{q \in \mathfrak{Q}}  \int_{\mathbf{w}} q(\mathbf{w})\textnormal{log}~\frac{p(\mathbf{y},\mathbf{w}|\mathbf{X},\mathbf{h})}{q(\mathbf{w})}d\mathbf{w}.
\]  
В данной работе в качестве множества $\mathfrak{Q}$ рассматривается нормальное распределение и распределение параметров, неявно получаемое оптимизацией градиентными методами. 

Оценка~\eqref{eq:var_elbo} является нижней, поэтому может давать некорректные оценки для правдоподобия~\eqref{eq:model_evidence}. Для того, чтобы оценить величину этой ошибки, докажем следующее утверждение.

\begin{theorem}[~\cite{Bishop}]\label{st:st1} Пусть задано множество $\mathfrak{Q} = \{q(\mathbf{w})\}$ непрерывных распределений. Максимизация вариационной нижней оценки $$\int_{\mathbf{w}} q(\mathbf{w})\textnormal{log}~\frac{p(\mathbf{y},\mathbf{w}|\mathbf{X},\mathbf{h})}{q(\mathbf{w})}d\mathbf{w}$$  логарифма интеграла~\eqref{eq:evidence}  эквивалентна минимизации расстояния Кульбака--Лейблера между распределением $q(\mathbf{w}) \in \mathfrak{Q}$ и апостериорным распределением параметров $p(\mathbf{w}|\mathbf{y}, \mathbf{X}, \mathbf{h})$:
\begin{equation}
\label{eq:optim}
    \hat{q} = \argmax_{q \in Q} \int_{\mathbf{w}} q(\mathbf{w})\textnormal{log}~\frac{p(\mathbf{y},\mathbf{w}|\mathbf{X},\mathbf{h})}{q(\mathbf{w})}d\mathbf{w} \Leftrightarrow 	
    \hat{q} = \argmin_{q \in Q} \textnormal{D}_\textnormal{KL}  \bigl(q(\mathbf{w})||p(\mathbf{w}|\mathbf{y}, \mathbf{X}, \mathbf{h})\bigr),
\end{equation}

\[
	\textnormal{D}_\textnormal{KL}  \bigl(q(\mathbf{w})||p(\mathbf{w}|\mathbf{y}, \mathbf{X}, \mathbf{h})\bigr) =  \int_\mathbf{w} q(\mathbf{w}) \frac{q(\mathbf{w})}{p(\mathbf{w}|\mathbf{y}, \mathbf{X}, \mathbf{h})} d\mathbf{w}.
\]

\end{theorem}
\begin{proof}
Доказательство непосредственно следует из~\eqref{eq:var_elbo}. Вычитая из обеих частей равенства $\textnormal{D}_\textnormal{KL}  (q(\mathbf{w})||p(\mathbf{w}|\mathbf{y}, \mathbf{X}, \mathbf{h}))$, получим
\[
\textnormal{log}~p(\mathbf{y}|\mathbf{X},\mathbf{h}) - \textnormal{D}_\textnormal{KL}  (q(\mathbf{w})||p(\mathbf{w}|\mathbf{y}, \mathbf{X}, \mathbf{h}))  = \int_{\mathbf{w}} q(\mathbf{w})\textnormal{log}~\frac{p(\mathbf{y},\mathbf{w}|\mathbf{X},\mathbf{h})}{q(\mathbf{w})}d\mathbf{w},
\]
где $\textnormal{log}~p(\mathbf{y}|\mathbf{X},\mathbf{h})$ --- выражение, не зависящее от $q(\mathbf{w})$.
\end{proof}



Таким образом, задача нахождения вариационной оценки, близкой к значению интеграла~\eqref{eq:model_evidence} сводится к поиску распределения $\hat{q}$, аппроксимирующего распределение $p(\mathbf{w}|\mathbf{y}, \mathbf{X}, \mathbf{h})$ наилучшим образом. 

\begin{defin} Пусть задано множество распределений $\mathfrak{Q}$. Модель $\mathbf{f}$ назовем субоптимальной на множестве моделей $M$, если модель доставляет максимум нижней вариационной оценке интеграла~\eqref{eq:optim}
\begin{equation}
\label{eq:var_elbo2}
	\max_{q \in \mathfrak{Q}}\int_{\mathbf{w}} q(\mathbf{w})\textnormal{log}~\frac{p(\mathbf{y},\mathbf{w}|\mathbf{X},\mathbf{h})}{q(\mathbf{w})}d\mathbf{w}.
\end{equation}
\end{defin}

{
Субоптимальность модели может быть также названа вариационной оптимальностью модели или LB-оптимальностью (\textit{Lower Bound --- нижняя граница}) модели.}

Вариационная оценка~\eqref{eq:var_elbo} интерпретируется как оценка сложности модели по принципу минимальной длины описания~\eqref{eq:mdl}, где первое слагаемое определяет количество информации для описания выборки, а второе слагаемое --- длину описания самой модели~\cite{nips}.

В данной работе решается задача выбора субоптимальной модели при различных заданных множествах $\mathfrak{Q}$.

\section{Методы получения вариационной оценки правдоподобия}
Ниже представлены методы получения вариационных нижних оценок~\eqref{eq:var_elbo2} правдоподобия~\eqref{eq:evidence}. В первом параграфе рассматривается метод, основанный на аппроксимации апостериорного распределения $p( \mathbf{w}|\mathbf{y}, \mathbf{X}, \mathbf{h})$~\eqref{eq:posterior} многомерным гауссовым распределением с диагональной матрицей ковариаций. В последующих параграфах рассматриваются методы, основанные на различных модификациях стохастического градиентного спуска. 

\textbf{Аппроксимация нормальным распределением. }
В качестве множества $\mathfrak{Q} = \{q(\mathbf{w})\}$ задано параметрическое семейство нормальных распределений с диагональными матрицами ковариаций:
\begin{equation}
\label{eq:diag}
	q \sim \mathcal{N}(\boldsymbol{\mu}_q, \mathbf{A}^{-1}_q),\quad,\boldsymbol{\theta}=[\boldsymbol{\mu}_q, \textbf{diag}(\mathbf{A}^{-1}_q)]
\end{equation}
где $\mathbf{A}_q$ --- диагональная матрица ковариаций, $\boldsymbol{\mu}_q$ --- вектор средних компонент.

Пусть априорное распределение $p(\mathbf{w}|\mathbf{h})$~\eqref{eq:prior} параметров модели задано как нормальное:
\[
	p(\mathbf{w}|\mathbf{h}) \sim \mathcal{N}(\boldsymbol{\mu}, \mathbf{A}^{-1}),\quad \mathbf{h} = \textbf{diag}(\mathbf{A}^{-1}_q),
\] 
Тогда оптимизация~\eqref{eq:optim} имеет вид
\begin{equation}
\label{eq:norm_max}
 \int_{\mathbf{w}} q(\mathbf{w})\textnormal{log}~{p(\mathbf{y}|\mathbf{X},\mathbf{w},\mathbf{h})} d \mathbf{w} - D_\textnormal{KL}\bigl(q (\mathbf{w} )|| p (\mathbf{w}|\mathbf{h})\bigr) \to \max_{\mathbf{A}_q, \boldsymbol{\mu}_q},
\end{equation}
где расстояние $D_\textnormal{KL}$ между двумя гауссовыми величинами рассчитывается как 
\[
	D_\textnormal{KL}\bigl(q (\mathbf{w}) || p (\mathbf{w}|\mathbf{h})\bigr) = \frac{1}{2} \bigl( \textnormal{Tr} [\mathbf{A}\mathbf{A}^{-1}_q] + (\boldsymbol{\mu} - \boldsymbol{\mu}_q)^\mathsf{T}\mathbf{A}(\boldsymbol{\mu} - \boldsymbol{\mu}_q) - u +\textnormal{ln}~|\mathbf{A}^{-1}| - \textnormal{ln}~|\mathbf{A}_q^{-1}| \bigr).
\]
В качестве приближенного значения интеграла $$\int_{\mathbf{w}} q(\mathbf{w})\textnormal{log}~{p(\mathbf{y}|\mathbf{X},\mathbf{w},\mathbf{h})} d \mathbf{w}$$ предлагается использовать формулу
\[
\int_{\mathbf{w}} q(\mathbf{w})\textnormal{log}~{p(\mathbf{y}|\mathbf{X},\mathbf{w},\mathbf{h})} d \mathbf{w} \approx \sum_{i=1}^m \textnormal{log}~p({y}_i|\mathbf{x}_i, \mathbf{w}_i),
\]
где $\mathbf{w}_i$  --- реализация случайной величины из распределения $q(\mathbf{w})$.

Итоговая функция оптимизации~\eqref{eq:norm_max} имеет вид
\begin{equation}
\label{eq:gaus}
	\mathbf{f} = \argmax_{\mathbf{A}_q, \boldsymbol{\mu}_q} \sum_{i=1}^m \textnormal{log}~p({y}_i|\mathbf{x}_i, \mathbf{w}_i) - D_\textnormal{KL}\bigl(q (\mathbf{w} )|| p (\mathbf{w}|\mathbf{h})\bigr).
\end{equation}

%\begin{example_empty} 
%\hspace{\parindent}
\begin{example}
Пусть  задана выборка $\mathfrak{D}$, в которой переменная ${y}$ не зависит от $\mathbf{x}$:
\begin{equation}
\label{eq:example_post}
	{y} \sim \mathcal{N}(\mathbf{w}, \mathbf{B}^{-1}),
\end{equation}

\[
	\mathbf{B}^{-1} = \left( \begin{array}{cc}
	2 & 1,8 \\
	1,8 & 2\\
	\end{array}  \right),
\]
\[
	p(\mathbf{w}|\mathbf{h}) = \mathcal{N}(\mathbf{0}, \mathbf{I}).
\]

График аппроксимации распределения параметров представлен на рис.~\ref{fig:var},\textit{а}. Как видно из графика, с использованием метода~\eqref{eq:gaus} получено грубое приближение апостериорного распределения $p(\mathbf{w}|\mathbf{y}, \mathbf{X}, \mathbf{h})$, что может существенно занизить оценку правдоподобия модели.


\begin{figure}[tbh!]



\minipage{0.32\textwidth}
 \caption*{\textit{а}}
  \includegraphics[width=\linewidth]{./plots/var/mf.pdf}

\endminipage\hfill
\minipage{0.32\textwidth}
\caption*{\textit{б}}
 
  \includegraphics[width=\linewidth]{./plots/var/sgd.pdf}
 \endminipage\hfill
\minipage{0.32\textwidth}%
 \caption*{\textit{в}}

  \includegraphics[width=\linewidth]{./plots/var/lang.pdf}
\endminipage\hfill
  \caption{Аппроксимация распределения \textit{а}) нормальным распределением, \textit{б}) распределением,
полученным с помощью градиентного спуска, \textit{в}) с использованием стохастической динамики Ланжевена.}
\label{fig:var}
\end{figure}




{Данный пример показывает, что качество итоговой аппроксимации распределения $p(\mathbf{w}|\mathbf{y}, \mathbf{X}, \mathbf{h})$ значительно зависит от схожести распределений $\hat{q}$ и $p(\mathbf{w}|\mathbf{y}, \mathbf{X}, \mathbf{h})$. В силу диагональности матрицы $\mathbf{A}_q$ и полного ранга матрицы $\mathbf{B}$  итоговое распределение $\hat{q}$ не может адекватно приблизить данное распределение  $p(\mathbf{w}|\mathbf{y}, \mathbf{X}, \mathbf{h})$.}

\end{example}

%\end{example_empty}
\textbf{Аппроксимация с использованием градиентного метода. }
В качестве множества распределений $\mathfrak{Q} = \{q(\mathbf{w})\}$, аппроксимирующих неизвестное распределение $\textnormal{log}~p(\mathbf{y}|\mathbf{X},\mathbf{h})$, используются распределения параметров, полученные в ходе их оптимизации. 

Представим неравенство~\eqref{eq:var_elbo}
\begin{equation}
\label{eq:var_elbo_entropy}
 \textnormal{log}~p(\mathbf{y}|\mathbf{X},\mathbf{h}) \geq \int_\mathbf{w} q(\mathbf{w})\textnormal{log}~\frac{p(\mathbf{y},\mathbf{w}|\mathbf{X}, \mathbf{h})}{q(\mathbf{w})}d\mathbf{w} =  \mathsf{E}_{q(\mathbf{w)}}\bigl(\textnormal{log~}p (\mathbf{y}, \mathbf{w}|\mathbf{X}, \mathbf{h})\bigr) - \mathsf{S}\bigl({q(\mathbf{w)}}\bigr),
\end{equation}
где $\mathsf{S}$ --- энтропия распределения:
\[
\mathsf{S}\bigl({q(\mathbf{w)}}\bigr) = - \int_{\mathbf{w}} q(\mathbf{w})\textnormal{log}~q(\mathbf{w})d\mathbf{w},
\]
$$p (\mathbf{y}, \mathbf{w}|\mathbf{X}, \mathbf{h}) = p (\mathbf{w}| \mathbf{h}) p (\mathbf{y}|\mathbf{X}, \mathbf{w}, \mathbf{h}),$$
$\mathsf{E}_{q(\mathbf{w)}}\bigl(\textnormal{log~}p (\mathbf{y}, \mathbf{w}|\mathbf{X}, \mathbf{h})\bigr)$ --- матожидание логарифма вероятности $\textnormal{log~}p (\mathbf{y}, \mathbf{w}|\mathbf{X}, \mathbf{h})$:
\[
	\mathsf{E}_{q(\mathbf{w)}}\bigl(\textnormal{log~}p (\mathbf{y}, \mathbf{w}|\mathbf{X}, \mathbf{h})\bigr) = \int_\mathbf{w} \textnormal{log~}p (\mathbf{y}, \mathbf{w}|\mathbf{X}, \mathbf{h}) q(\mathbf{w}) d\mathbf{w}.
\]

Оценка распределений производится при оптимизации параметров. Оптимизация выполняется в режиме мультистарта~\cite{multi}, т.е. при запуске оптимизации параметров модели из нескольких разных начальных приближений. Основная проблема такого подхода~---~вычисление энтропии $\mathsf{S}$ распределений $q(\mathbf{w}) \in Q$. Ниже представлен метод получения оценок энтропии~\eqref{eq:entropy} ~$\mathsf{S}$ и оценок правдоподобия~\eqref{eq:var_elbo_entropy}.

Запустим $r$ процедур оптимизаций модели $\mathbf{f}$ из разных начальных приближений:
\[
	L = -\sum_{l=1}^r \text{log}p(\mathbf{y}, \mathbf{w}^l|\mathbf{X}, \mathbf{h})  \to \min,
\] 
где $r$ --- число оптимизаций,
\begin{equation}
\label{eq:loss_func}
\text{log}p(\mathbf{y}, \mathbf{w}^l|\mathbf{X}, \mathbf{h}) = -\sum_{i=1}^m \textnormal{log}p({y}_i, \mathbf{w}^l |\mathbf{x}_i, \mathbf{h}) = -\textnormal{log}~p(\mathbf{w}^l|\mathbf{h}) - \sum_{i=1}^m \textnormal{log}p({y}_i |\mathbf{x}_i, \mathbf{w}^l, \mathbf{h}).
\end{equation}

Пусть начальные приближения параметров $\mathbf{w}^1, \dots, \mathbf{w}^r$ порождены из некоторого начального распределения $q^0(\mathbf{w})$:
\[ 
	\mathbf{w}^1, \dots, \mathbf{w}^r \sim q^0(\mathbf{w}). 
\]

Параметры $\mathbf{w}^1, \dots, \mathbf{w}^r$ задают вариационное распределение $q$: $$\boldsymbol{\theta} = [\mathbf{w}^1, \dots, \mathbf{w}^r].$$
%Обозначим за  $\mathbf{w}^l, g \in \{1,\dots,r\}$ значения параметров $\mathbf{w}^1, \dots, \mathbf{w}^r$ на  текущем шаге оптимизации. 


Для дальнейшего описания метода введем понятие оператора градиентного спуска, являющегося частным случаем оператора оптимизации~\eqref{eq:optim_operator}.
\begin{defin}
Оператором градиентного спуска назовем оператор оптимизации вида
\begin{equation}
\label{eq:sgd}
	T(\boldsymbol{\theta}) = \boldsymbol{\theta} - \beta \nabla L, 
\end{equation}
где  $\beta$ --- длина шага градиентного спуска.
\end{defin}

Пусть значения $\mathbf{w}^1, \dots, \mathbf{w}^r$  --- реализации случайной величины из некоторого распределения $q(\mathbf{w})$. Начальная энтропия распределения $q(\mathbf{w})$ соответствует энтропии распределения $q^0(\mathbf{w})$, из которого были порождены начальные приближения оптимизации параметров $\mathbf{w}^1, \dots, \mathbf{w}^r$. Под действием оператора $T$ распределение параметров $\mathbf{w}_1, \dots, \mathbf{w}_r$ изменяется. Для учета энтропии распределений, полученных в ходе оптимизации,
{ формализуем метод,  представленный в~\cite{early}. }

\begin{theorem}~Пусть $T$ --- оператор градиентного спуска,
 $L$ --- функция потерь, градиент $\nabla L$ которой имеет константу Липшица $C_L$.  Пусть $\mathbf{w}^1,\dots,\mathbf{w}^r$ ---  начальные приближения оптимизации модели, где $r$ --- число начальных приближений. Пусть $\beta$ --- длина шага градиентного спуска, такая что
\begin{equation}
\label{eq:ineq}
\beta<\frac{1}{C_L}, \quad \beta < \bigl(\max_{g \in \{1,\dots,r\}}\lambda_\textnormal{max} (\mathbf{H}(\mathbf{w}^l))\bigr)^{-1}, 
\end{equation}
где $\lambda_\textnormal{max}$ --- наибольшее по модулю собственное значение гессиана  $\mathbf{H}$ функции потерь $L$.

При выполнении неравенств~\eqref{eq:ineq} разность энтропий распределений $q'(\mathbf{w}), q(\mathbf{w})$ на смежных шагах почти наверное сходится к следующему выражению: 
\begin{equation}
\label{eq:entropy}
	\mathsf{S}\bigl(q'(\mathbf{w})) -  \mathsf{S}\bigl(q(\mathbf{w}))  \approx  \frac{1}{r}\sum_{l=1}^r \bigl(-\beta \textnormal{Tr}[\mathbf{H}(\mathbf{w}'^l)] - \beta \textnormal{Tr}[\mathbf{H}(\mathbf{w}'^l)\mathbf{H}(\mathbf{w}'^l)]  \bigr) + o_{\beta^2 \to 0}(1),
\end{equation}
где $\mathbf{H}$ --- гессиан функции потерь $L$.
\end{theorem}



\begin{proof}
Предварительно приведем две леммы, требуемые для доказательства теоремы.
\begin{lemma}[~\cite{sgd_conv}] Пусть $T$ --- оператор градиентного спуска, $L$ --- дважды дифференцируемая функция потерь, градиент $\nabla L$ которой имеет константу Липшица $C_L$.  Пусть для длины шага $\beta$ выполнено неравенство 
$
	\beta<\frac{1}{C_L}.
$
Тогда $T$ является диффеоморфизмом.
\end{lemma}

\begin{lemma}[~\cite{entropy}] Пусть $\mathbf{w}$ --- случайный вектор с непрерывным распределением $q(\mathbf{w})$. Пусть $T$ --- биективное отображение вектора $\mathbf{w}$ в пространство той же размерности. Пусть $q'(\mathbf{w})$ --- распределение вектора $T(\mathbf{w})$. Тогда справедливо утверждение
\begin{equation}
\label{eq:entropy_biject}
	\mathsf{S}\bigl(q'(\mathbf{w})\bigr) -  \mathsf{S}\bigl(q(\mathbf{w})\bigr)  = \int_\mathbf{w}  q'(\mathbf{w}) \textnormal{log}~\left|\frac{\partial{T(\mathbf{w})}}{\partial{\mathbf{w}}}\right| d\mathbf{w}.
\end{equation}
\end{lemma}



Рассмотрим очередной шаг оптимизации. При $\beta<\frac{1}{C}$ оператор градиентного спуска $T$ является диффеоморфизмом, а значит, и биекцией, справедлива формула~\eqref{eq:entropy_biject}.
По усиленному закону больших чисел 
\[
	\mathsf{S}\bigl(q'(\mathbf{w})\bigr) -  \mathsf{S}\bigl(q(\mathbf{w})\bigr)  \approx  \frac{1}{r}\sum_{l=1}^r \textnormal{log}~\left|\frac{\partial{T(\mathbf{w}'^l)}}{\partial{\mathbf{w}}}\right|.
\]
Логарифм якобиана  $\textnormal{log}~\left|\frac{\partial{T(\mathbf{w}'^l)}}{\partial{\mathbf{w}}}\right|$ оператора $T$ запишем как%~\cite{early}:
\begin{equation}
\label{eq:to_taylor}
	\textnormal{log}~\left|\frac{\partial{T(\mathbf{w}'^l)}}{\partial{\mathbf{w}}}\right| = \textnormal{log}~|\mathbf{I} - \beta\mathbf{H}| = \sum_{i=1}^{u} \textnormal{log}~(1-\beta\lambda_i),
\end{equation}
где $\lambda_i$ --- $i$-е собственное значение гессиана $\mathbf{H}$.

При $(\beta\lambda_i) ^ 2 \leq (\beta\lambda_\textnormal{max})^2 < 1$ выражение~\eqref{eq:to_taylor} раскладывается в ряд Тейлора:
\[
	 \sum_{t=1}^{u} \textnormal{log}~(1-\beta\lambda_i) =  -\beta \textnormal{Tr}[\mathbf{H}(\mathbf{w}'^l)] - \beta^2 \textnormal{Tr}[\mathbf{H}(\mathbf{w}'^l)\mathbf{H}(\mathbf{w}'^l)] + o_{\beta^2 \to 0}(1).
\]
Просуммировав полученные выражения для каждой точки мультистарта и вынеся $o_{\beta^2 \to 0}(1)$ за скобки, получим выражение~\eqref{eq:entropy}, что и требовалось доказать.

\end{proof} 	


Получим итоговую формулу для оценки правдоподобия модели.
\begin{theorem}\label{st:st2}
Оценка~\eqref{eq:var_elbo_entropy} на шаге оптимизации $\tau$ представима в виде
\begin{equation}
\label{eq:ev_grad_full}
\textnormal{log}~\hat{p}(\mathbf{y}|\mathbf{X}, \mathbf{h}) \approx \frac{1}{r} \sum_{g = 1}^r L(\mathbf{w}^l_\tau, \mathbf{X}, \mathbf{y})  + 
\end{equation}
\[
+\mathsf{S}\big(q^0(\mathbf{w})\bigr) + \frac{1}{r}\sum_{b=1}^\tau\sum_{l=1}^r \bigl(-\beta \textnormal{Tr}[\mathbf{H}(\mathbf{w}_b^l)] - \beta^2 \textnormal{Tr}[\mathbf{H}(\mathbf{w}_b^l)\mathbf{H}(\mathbf{w}_b^l)]  \bigr) 
\]
с точностью до слагаемых вида $o_{\beta^2 \to 0}(1)$,
где $\mathbf{w}_b^l$ --- $l$-я реализация параметров модели на шаге оптимизации $b$, $q^0(\mathbf{w})$ --- начальное распределение.
\end{theorem}



\begin{proof} Представим энтропию распределения $q^\tau(\mathbf{w})$ следующим образом:
\[
\mathsf{S}\bigl(q^\tau(\mathbf{w})\bigr) = \mathsf{S}\bigl(q^0(\mathbf{w})\bigr) - \mathsf{S}\bigl(q^0(\mathbf{w})\bigr) + \mathsf{S}\bigl(q^1(\mathbf{w})\bigr) - \mathsf{S}\bigl(q^1(\mathbf{w})\bigr) +\dots -
\mathsf{S}\bigl(q^{\tau-1}(\mathbf{w})\bigr) + \mathsf{S}\bigl(q^\tau(\mathbf{w})\bigr).
\]
Каждая разность энтропий вида $\mathsf{S}\bigl(q^b(\mathbf{w})\bigr) - \mathsf{S}\bigl(q^{b-1}(\mathbf{w})\bigr)$ по теореме с точностью до $o_{\beta^2 \to 0}(1)$ представима в виде
\begin{equation}
\label{eq:eq_sums}
	\mathsf{S}\bigl(q^b(\mathbf{w})\bigr) -  \mathsf{S}\bigl(q^{b-1}(\mathbf{w})\bigr)  \approx  \frac{1}{r}\sum_{l=1}^r \bigl(-\beta \textnormal{Tr}[\mathbf{H}(\mathbf{w}_b^l)] - \beta^2 \textnormal{Tr}[\mathbf{H}(\mathbf{w}_b^l)\mathbf{H}(\mathbf{w}_b^l)]  \bigr).
\end{equation}

Формула~\eqref{eq:ev_grad_full} получается подстановкой в выражение~\eqref{eq:var_elbo_entropy} суммы выражений вида~\eqref{eq:eq_sums}, а также начальной энтропии $\mathsf{S}\bigl(q^0(\mathbf{w}))$.
\end{proof}

В~\cite{early} предлагается алгоритм приближенного вычисления для выражения, находящегося под знаком суммы в~\eqref{eq:ev_grad_full}:
\[
	-\beta \textnormal{Tr}[\mathbf{H}(\mathbf{w}^l)] - \beta^2 \textnormal{Tr}[\mathbf{H}(\mathbf{w}^l)\mathbf{H}(\mathbf{w}^l)]  \approx \mathbf{r}_0^\mathsf{T}\bigl(-2\mathbf{r}_0 + 3\mathbf{r}_1 -\mathbf{r}_2\bigr),
\]
где вектор $\mathbf{r}_0$  порождается из нормального распределения:
$$\mathbf{r}_0 \sim \mathcal{N}(\mathbf{0}, \mathbf{I}), \quad \mathbf{r}_1 = \mathbf{r}_0 - \beta \mathbf{r}_0^\mathsf{T} \nabla \nabla L, \quad \mathbf{r}_2 = \mathbf{r}_1 - \beta \mathbf{r}_1^\mathsf{T} \nabla \nabla L.$$


Заметим, что при приближении параметров модели к точке экстремума оценка правдоподобия устремляется в минус бесконечность в силу постоянно убывающей энтропии. Таким образом, чем ближе градиентный метод приближает параметры модели к точке экстремума, тем менее точной становится оценка правдоподобия модели. Один из методов борьбы с данной проблемой представлен в следующих параграфах.
\begin{figure}
\caption{Псевдокод алгоритма получения вариационной нижней оценки правдоподобия модели с использованием градиентного спуска}
\label{fig:algo}
\begin{algorithmic}[1]
\REQUIRE $\mathbf{X}, \mathbf{y}, p(\mathbf{w}|\mathbf{h})$;
\REQUIRE критерий останова $\text{Stop}$, начальное распределение параметров $q^0$, количество точек мультистарта $r$, функция потерь $L$, ее первая и вторая производные;
\ENSURE $\textnormal{log}~\hat{p}(\mathbf{y}|\mathbf{X}, \mathbf{h})$;
\FOR{$l=1,\dots,r$}
\STATE $\mathbf{w}^l \sim q^0$;
\ENDFOR
\STATE $\mathsf{S} = \mathsf{S}\bigl(q^0)$;
\WHILE{не достигнут критерий останова $\text{Stop}$}
\STATE $\boldsymbol{\theta} = T(\boldsymbol{\theta});$
\FOR{$l=1,\dots,r$}
\STATE $\mathbf{r}_0 \sim \mathcal{N}(\mathbf{0}, \mathbf{I})$;
\STATE $\mathbf{r}_1 = \mathbf{r}_0 - \beta \mathbf{r}^{\mathsf{T}}_0 \nabla \nabla L(\mathbf{w}^l, \mathbf{y}, \mathbf{X})$;
\STATE $\mathbf{r}_2 = \mathbf{r}_1 - \beta \mathbf{r}^{\mathsf{T}}_1 \nabla \nabla L(\mathbf{w}^l, \mathbf{y}, \mathbf{X})$;
\STATE $\mathsf{S}^l = \mathbf{r}_0^\mathsf{T}\bigl(-2\mathbf{r}_0 + 3\mathbf{r}_1 -\mathbf{r}_2\bigr)$;
\ENDFOR
\STATE $\mathsf{S} = \frac{1}{r}\sum_{l=1}^r \mathsf{S}^l$;
\ENDWHILE
\STATE $\hat{p}(\mathbf{y}|\mathbf{X}, \mathbf{w}, \mathbf{h}) = \frac{1}{r}\sum_{l=1}^r p(\mathbf{y}|\mathbf{X}, \mathbf{w}^l, \mathbf{h})$;
\STATE $\hat{p}(\mathbf{w} | \mathbf{h}) = \frac{1}{r}\sum_{l=1}^r p(\mathbf{w}^l| \mathbf{h})$;
\STATE $\textnormal{log}~\hat{p}(\mathbf{y}|\mathbf{X}, \mathbf{h}) = \textnormal{log}~\hat{p}(\mathbf{y}|\mathbf{X}, \mathbf{w}, \mathbf{h}) +\textnormal{log}~\hat{p}(\mathbf{w} | \mathbf{h})$;
	
\end{algorithmic}
\end{figure}



\textbf{Модификация алгоритма оптимизации модели.} \\
В качестве оператора $T$ предлагается использовать псевдослучайный стохастический градиентный спуск, т.е. градиентный спуск~\eqref{eq:sgd_operator}, оптимизирующий параметры $\mathbf{w}^1,\dots,\mathbf{w}^r$ по некоторой случайной подвыборке $\hat{\mathbf{X}}, \hat{\mathbf{y}}$, одинаковой для каждой точки старта $\mathbf{w}^1,\dots,\mathbf{w}^r$:
\[
    T( \boldsymbol{\theta}| L,\mathbf{X},  \mathbf{y},  \mathbf{h}, \boldsymbol{\beta}) = \boldsymbol{\theta} - \beta\nabla L(\boldsymbol{\theta},   \mathbf{h},  \hat{\mathbf{X}}, \hat{\mathbf{y}}),
\]
где $\beta_{\text{lr}}$ --- шаг градиентного спуска, $\hat{\mathbf{y}}, \hat{\mathbf{X}}$ --- случайная подвыборка заданной мощности выборки $\mathfrak{D}$.

где $\hat{\mathbf{X}}$ --- случайная подвыборка выборки ${\mathbf{X}}$, одинаковая для всех точек мультистарта, $\hat{\mathbf{y}}$ --- соответствующие метки классов, $$|\hat{\mathbf{X}}| = \hat{m}.$$

Как и версия алгоритма с использованием градиентного спуска~\eqref{eq:sgd}, основной проблемой модифицированного алгоритма оценки интеграла~\eqref{eq:var_elbo2} является грубость аппроксимации исходного распределения $p(\mathbf{w}|\mathbf{f},\mathfrak{D})$.

Рассмотрим пример~\eqref{eq:example_post}.
График аппроксимации распределения $p(\mathbf{w}|\mathbf{y}, \mathbf{X}, \mathbf{h})$ представлен на рис.~\ref{fig:var},\textit{б}.
Как видно из графика, градиентный спуск сходится к моде распределения. При небольшом количестве итераций полученное распределение также слабо аппроксимирует апостериорное распределение. {При приближении к точке экстремума снижается вариационная оценка правдоподобия модели, что  интерпретируется как возможное начало переобучения~\cite{early}. Таким образом, снижение оценки~\eqref{eq:ev_grad_full} можно использовать как критерий остановки оптимизации модели для снижения эффекта переобучения.  }

На рис.~\ref{fig:var} представлена  {аппроксимация распределения $p(\mathbf{w}|\mathbf{Y}, \mathbf{X}, \mathbf{h})$ различными методами: \textit{а}) нормальным распределением с диагональной матрицей ковариаций, \textit{б}) с помощью градиентного спуска, \textit{в}) с помощью стохастической динамики Ланжевена. Точками отмечены параметры модели $\mathbf{f}$, полученные в ходе нескольких запусков оптимизации и являющиеся реализациями случайной величины с распределением $q(\mathbf{w})$. Нормальное распределение слабо аппроксимирует распределение $p(\mathbf{w}|\mathbf{Y}, \mathbf{X}, \mathbf{h})$ в силу диагональности матрицы ковариаций. Распределение, полученное с помощью градиентного спуска, слабо аппроксимирует распределение $p(\mathbf{w}|\mathbf{Y}, \mathbf{X}, \mathbf{h})$, так как сходится к моде.}





\textbf{Аппроксимация с использованием динамики Ланжевена}\\
Для достижения нижней оценки интеграла~\eqref{eq:var_elbo2}, более близкой к реальному значению логарифма интеграла~\eqref{eq:evidence}, чем оценка с использованием градиентного спуска, предлагается использовать стохастическую динамику Ланжевена~\cite{langevin}. Стохастическая динамика Ланжевена представляет собой вариант стохастического градиентного спуска с добавлением гауссового шума:
\begin{equation}
\label{eq:langevin}
	T(\mathbf{w}) = \mathbf{w} -  \beta \nabla L -\frac{m}{\hat{m}}\textnormal{log}p(\hat{\mathbf{y}}|\hat{\mathbf{X}}, \mathbf{w},\mathbf{h}) + \boldsymbol{\varepsilon}, \quad  \boldsymbol{\varepsilon} \sim \mathcal{N}(\mathbf{0}, {\frac{\beta}{2}}\mathbf{I}),
\end{equation}
где $\hat{\mathbf{X}}$ --- псевдослучайная подвыборка, $\hat{\mathbf{y}}$ --- соответствующие метки, $\hat{m}$ --- размер подвыборки. Длина шага оптимизации $\beta$ удовлетворяет  {условиям, гарантирующим сходимость алгоритма в стандартных ситуациях~\cite{langevin}}:
\[
	\sum_{\tau=1}^\infty \beta_\tau = \infty, \quad \sum_{\tau=1}^\infty \beta_\tau^2 < \infty.
\]

Для оценки энтропии с учетом шума $\boldsymbol{\varepsilon}$ предлагается использовать следующее неравенство~\cite{entropy,var_grad}:
\[
\hat{\mathsf{S}}\bigl(q^\tau(\mathbf{w})\bigr)   \geq \frac{1}{2}u\textnormal{log}\left(\textnormal{exp}\left(\frac{2\mathsf{S}\bigl(q^\tau(\mathbf{w})\bigr)}{u}\right) + \textnormal{exp}\left(\frac{2\mathsf{S}\bigl( \boldsymbol{\varepsilon})}{u}\right)\right),
\]
{где  $\tau$ --- текущий шаг оптимизации,} $\mathsf{S}\bigl( \mathcal{N}({0}, {\frac{\beta}{2}})\bigr)$ --- энтропия нормального распределения, $\hat{\mathsf{S}}(q^\tau(\mathbf{w}))$ --- энтропия распределения $q^\tau$ с учетом добавленного шума~$\boldsymbol{\varepsilon}$.


В отличие от стохастического градиентного спуска стохастическая динамика Ланжевена сходится к апостериорному распределению параметров $p(\mathbf{w}|\mathfrak{D},\mathbf{h})$~\cite{langevin, langevin_sato}.  График аппроксимации апостериорного распределения с использованием динамики Ланжевена представлен на рис.~\ref{fig:var},\textit{в}. При одинаковом количестве итераций динамика Ланжевена продолжает аппроксимировать апостериорное распределение, в то время как градиентный спуск сходится к моде распределения. {Как видно из графика, алгоритм, основанный на стохастической динамике Ланжевена, способен давать более точную вариационную оценку правдоподобия~\eqref{eq:var_elbo2}. В то же время алгоритм более требователен к настройке параметров оптимизации~\cite{sgld}: \textit{``быстро изменяющаяся кривизна [траекторий параметров модели] делает методы стохастической градиентной динамики Ланжевена по умолчанию неэффективными''.}}
%However, the rapidly changing curvature renders default SGLD methods inefficient




\section{Анализ методов выбора моделей}
Для анализа свойств предложенного критерия субоптимальности в задачах регрессии и классификации, а также методов получения нижних оценок правдоподобия модели в задачах выбора моделей был проведен ряд вычислительных экспериментов на выборках Boston Housing, Protein Structure, а также на небольшой подвыборке YearPredictionMSD (далее --- Boston, Protein и MSD)~\cite{UCI} {и подвыборке изображений рукописных цифр MNIST~\cite{mnist}}.

{Для выборок Boston, Protein и MSD} была рассмотрена задача регрессии
\[
	\mathbf{y} = \mathbf{f}(\mathbf{X}, \mathbf{w}) + \boldsymbol{\varepsilon}, \quad  \boldsymbol{\varepsilon} \sim \mathcal{N}(\mathbf{0}, \mathbf{I}), \mathbf{f} \in {M}.
\]

В качестве множества моделей $M$ были рассмотрены  нейросети с одним скрытым слоем и softplus-функцией активации:
\begin{equation}
\label{eq:model}
	\mathbf{f}(\mathbf{w}, \mathbf{X}) =   \textbf{softplus}\bigl(\mathbf{X} \mathbf{W}_1 \bigr)  \mathbf{W}_2,
\end{equation}
где $\mathbf{W}_1 \in \mathbb{R}^{n\times n_1}$ --- матрица параметров скрытого слоя нейросети, $\mathbf{W}_2 \in \mathbb{R}^{n_1\times 1}$ --- матрица параметров выходного слоя нейросети, {$\textbf{softplus}(\mathbf{X}) = \textbf{log}\bigl(1+\textbf{exp}(\mathbf{X})\bigr)$}.

{Для выборки Boston также было рассмотрено множество моделей с тремя скрытыми слоями, построенных аналогично однослойной модели~\eqref{eq:model}. Размер каждого слоя равнялся 50.}

{Для выборки MNIST была рассмотрена задача бинарной классификации: из выборки были взяты только объекты, соответствующие цифрам 7 и 9. Размерность выборки была понижена с 784 до 50 методом главных компонент аналогично~\cite{firefly}. Для анализа моделей, полученных в случае высокой вероятности переобучения, из обучающей выборки были взяты первые 500 объектов. В качестве модели рассматривалась нейросеть с тремя скрытыми слоями}
\[
    \mathbf{f}(\mathbf{w}, \mathbf{X}) =   \boldsymbol{\sigma}(\textbf{softplus}\bigl(  \textbf{softplus} \bigl(\textbf{softplus}\bigl(\mathbf{X} \mathbf{W}_1 \bigr)  \mathbf{W}_2 \bigr) \mathbf{W}_3 \bigr) \mathbf{W}_4),
\]
{где $\boldsymbol{\sigma}(\mathbf{X}) = \bigl(1+\textbf{exp}(\mathbf{-X})\bigr)^{-1}$  --- сигмоида, $\mathbf{W}_1, \dots, \mathbf{W}_4$ --- параметры нейросети.}



Во всех экспериментах исходная выборка $\mathfrak{D}$ разбивалась на обучающую и контрольную подвыборки:
$
	\mathfrak{D} = \mathfrak{D}_\textnormal{train} \sqcup \mathfrak{D}_\textnormal{text}.
$

Оптимизация параметров производилась на подвыборке $\mathfrak{D}_\textnormal{train}$. Для контроля переобучения некоторых алгоритмов из обучающей выборки $\mathfrak{D}_\textnormal{train}$ формировалась валидационная выборка $\mathfrak{D}_\textnormal{valid}$, на которой не проводилась оптимизация параметров  модели. Мощность валидационной выборки $\mathfrak{D}_\textnormal{valid}$ составляла 0,1 мощности обучающей выборки  $\mathfrak{D}_\textnormal{train}$, объекты для валидационной выборки выбирались случайным образом независимо для каждого старта алгоритма.
Качество полученных моделей проверялось на подвыборке $\mathfrak{D}_\textnormal{test}.$ Критерием качества модели выступали среднеквадратичное отклонение вектора $\mathbf{y}$ от вектора $\mathbf{f}(\mathbf{w}, \mathbf{X})$ (RMSE) {в случае задачи регрессии и доля верно предсказанных меток класса (Accuracy) в задаче классификации}, а также { соответствующие критерии } при возмущении элементов выборки:
\begin{equation}
\label{eq:rmse}
	\textnormal{RMSE}_{\sigma} =\textnormal{RMSE}\bigl(\mathbf{f}( \mathbf{w}, \mathbf{X}+\boldsymbol{\varepsilon}), \mathbf{y}\bigr),  \quad \boldsymbol{\varepsilon} \sim \mathcal{N}(\mathbf{0}, \sigma \mathbf{I}).
\end{equation}

Были рассмотрены шесть алгоритмов.
\begin{enumerate}
\item Базовый алгоритм: оптимизация параметров без валидации и ранней остановки. Оптимизация проводилась с использованием стохастического градиентного спуска~\eqref{eq:sgd}. Для данного алгоритма априорное распределение $p(\mathbf{w}|\mathbf{h})$ не использовалось.
\item Алгоритм с валидацией. Для контроля переобучения во время оптимизации качество модели оценивалось на валидационной выборке $\mathfrak{D}_\textnormal{valid}$. Для данного алгоритма априорное распределение также не использовалось.
\item Алгоритм с валидацией и введенным априорным распределением. В качестве априорного распределения рассматривается распределение вида
$
	\mathbf{w} \sim \mathcal{N}(\mathbf{0}, \alpha \mathbf{I}), 
$
где $\alpha$ --- дисперсия.

\item Нахождение вариационной нижней оценки с использованием стохастического градиентного спуска.
\item Нахождение вариационной нижней оценки с использованием стохастической динамики Ланжевена.
\item Нахождение вариационной нижней оценки с аппроксимацией нормальным распределением ~\eqref{eq:gaus}.
\end{enumerate}




Параметры модели выбирались из точек мультистарта (алгоритмы 1---5) или порождались из распределения $\hat{q}$ (алгоритм 6). Количество точек мультистарта: $r=10$ {для задач регрессии и $r=25$ для задачи классификации}.
Для алгоритмов 2---6 применялась ранняя остановка: каждые $\tau_\textnormal{val}$ итераций производилась оценка внутреннего критерия качества модели. В качестве критерия остановки применялось следующее условие: значение внутреннего критерия качества не улучшалось $3\tau_\textnormal{val}$ итераций. Для разных алгоритмов внутренним критерием качества выступали различные величины:
\begin{enumerate}
\item функция потерь $L$~\eqref{eq:loss_func} на валидационной выборке $\mathfrak{D}_\textnormal{valid}$ для алгоритмов $2,3$,
\item вариационная нижняя оценка правдоподобия~\eqref{eq:var_elbo} на обучающей выборке $\mathfrak{D}_\textnormal{train}$ для алгоритмов $4,5,6$.
\end{enumerate}

Для каждой модели назначались различные значения параметра $\alpha (\alpha \in \{10, \dots, 10^9\})$ и длины шага оптимизации $\beta$, отбирались наилучшие модели. 



Описание эксперимента представлено в табл.~1. Результаты экспериментов представлены в табл.~2. На рис.~\ref{fig:noise_in_data} представлен график зависимости $\textnormal{RMSE}_{\sigma}$ от параметра~$\sigma$~{ для однослойных моделей}. 

\begin{figure}[tbh!]


\minipage{0.32\textwidth}
\caption*{\textit{а}}
\includegraphics[width=1.0\textwidth]{./plots/var/boston/rmse_data2.pdf}
\endminipage\hfill
\minipage{0.32\textwidth}
\caption*{\textit{б}}
\includegraphics[width=1.0\textwidth]{./plots/var/protein/rmse_data2.pdf}

\endminipage\hfill
\minipage{0.32\textwidth}%
\caption*{\textit{в}}
\includegraphics[width=1.0\textwidth]{./plots/var/msd/rmse_data2.pdf}

\endminipage
\caption{ Возмущение выборки для однослойных нейросетей: \textit{а}) Boston Housing, \textit{б}) Protein, \textit{в}) MSD.
}
\label{fig:noise_in_data}
\end{figure}




\begin{table}[!htbp]
\captionsetup{justification=raggedright,singlelinecheck=false}
\label{table1}
\caption{Описание выборок для экспериментов  по выбору моделей}
\footnotesize
\centering

\begin{tabular}{ | p{2cm} |p{2cm} | p{2cm} | p{2cm} | p{2cm} | p{2cm} | }
\hline
Выборка $\mathfrak{D}$ & Интервал валидации, $\tau_\textnormal{val}$ & Количество объектов, $m$ & Количество признаков, $n$ & Размер подвыборки, $\hat{m}$ &  Размер скрытого слоя, $n_1$ \\
\hline
Boston Housing & 100 & 506 & 13 & $\hat{m} = m$ & 50 \\
\hline
Protein & 1000 & 45000 & 9 & $\hat{m} = 200$ & 100 \\
\hline
MSD & 1000& 5000 & 91 & $\hat{m} = 50$ & 100\\
\hline
MNIST & 100  & 500 & 50 & $\hat{m} = 100$ & 50\\ 
\hline
\end{tabular}
\end{table}


\newcommand{\specialcell}[2][c]{%
  \begin{tabular}[#1]{@{}c@{}}#2\end{tabular}}

\begin{table}[htbp!]
\captionsetup{justification=raggedright,singlelinecheck=false}
\centering
\label{table2}
\caption{Результаты эксперимента по выбору моделей}
\footnotesize
\begin{tabular}{ | c | c | c | c | c | c | c |}

\hline
&\multicolumn{6}{|c|}{Алгоритмы}  \\
\hline
Выборка $\mathfrak{D}$ & 1 & 2 & 3 & 4 & 5 & 6 \\
\hline
\multicolumn{7}{|c|}{Результаты, RMSE/Accuracy}  \\

\hline
\specialcell{ Boston,  \\один  слой} & 8,1  $\pm$ 2,0 & 5,9 $\pm$ 0,7 & 5,2 $\pm$ 0,6 & $\mathbf{3,7 \pm 0,2}$ & 6,7 $\pm$ 0,7 & 5,0 $\pm$ 0,4 \\
\hline
Boston, 3 слоя & 7,1 $\pm$ 1,3 & 4,3 $\pm$ 0,1 & 4,4 $\pm$ 0,4 & $\mathbf{3,2 \pm 0,06}$ & 4,6 $\pm$ 0,4 & 6,8 $\pm$ 1,6 \\
\hline
Protein & 5,1 $\pm$ 0,0 & 5,1 $\pm$ 0,0 & 5,1 $\pm$ 0,0 & 5,1 $\pm$ 0,0 & 5,1 $\pm$ 0,0 & $\mathbf{5,0 \pm 0,1}$ \\
\hline
MSD & 12,2 $\pm$ 0,0 & $\mathbf{10,9 \pm 0,1}$ & $\mathbf{10,9 \pm 0,1}$ & 12,2 $\pm$ 0,0 & 12,9 $\pm$ 0,0 & 19,6 $\pm$ 3,6  \\
\hline
MNIST & 0,985 $\pm$ 0,002 & 0,984 $\pm$ 0,002 & $\mathbf{0,986 \pm 0,002}$ & 0,914 $\pm$ 0,005 & 0,979 $\pm$ 0,003 & 0,971 $\pm$ 0,001 \\
\hline

\multicolumn{7}{|c|}{Результаты, $\textnormal{RMSE/Accuracy}_{0,5}$}  \\
\hline
\specialcell{ Boston,  \\один  слой} & 43,9  $\pm$ 9,4 & 18,6 $\pm$ 2,0 &  15,8 $\pm$ 2,3 & $\mathbf{11,9 \pm 1,1}$ & 20,3 $\pm$ 3,1 & 18,2 $\pm$ 3,3 \\
\hline
Boston, 3 слоя & 23,4 $\pm$ 4,9 & 18,7 $\pm$ 2,8 & 18,3 $\pm$ 3,0 & \bf 9,0 $\pm$ 0,7 & 14,5 $\pm$ 2,6 &  15,2 $\pm$ 2,7 \\
\hline
Protein & 19,5 $\pm$ 0,3 & 18,5 $\pm$ 0,5 & 18,6 $\pm$ 0,3 & $\mathbf{16,7 \pm 0,3}$ & 19,3 $\pm$ 0,6 & 19,7 $\pm$ 3,7  \\
\hline
MSD & 178,3 $\pm$ 0,8 & $\mathbf{121,3 \pm 4,5}$ & 123,7 $\pm$ 2,5 & 175,8 $\pm$ 1,0 & 203,8 $\pm$ 1,4 & 292,0 $\pm$ 2,0 \\
\hline
MNIST & 0,931 $\pm$ 0,004 & 0,929 $\pm$ 0,006 & $\mathbf{0,934 \pm 0,007}$ & 0,857 $\pm$ 0,007 & 0,919 $\pm$ 0,008 & 0,916 $\pm$ 0,004 \\
\hline


\multicolumn{7}{|c|}{Результаты, $\textnormal{RMSE/Accuracy}_{1,0}$}  \\
\hline
\specialcell{ Boston,  \\один  слой} & 120,9 $\pm$ 33,4 & 42,5 $\pm$ 6,3 & 32,5 $\pm$ 6,0 & $\mathbf{25,7 \pm 3,2}$ & 42,4 $\pm$ 5,7 & 41,3 $\pm$ 6,3  \\
\hline
Boston, 3 слоя & 46,1 $\pm$ 15,8 & 40,5 $\pm$ 5,3 & 38,6 $\pm$ 8,0 & \bf 16,5 $\pm$ 2,5 & 30,4 $\pm$ 7,9 & 26,2 $\pm$ 6,9 \\
\hline
Protein & 37,0 $\pm$ 0,8 & 34,4 $\pm$ 1,1 & 35,0 $\pm$ 1,0 & $\mathbf{30,6 \pm 0,6}$ & 36,6 $\pm$ 1,1 & 35,0 $\pm$ 8,1 \\
\hline
MSD & 319,6 $\pm$ 1,4 & $\mathbf{217,5 \pm 8,2}$ & 221,9 $\pm$ 4,2 & 314,8 $\pm$ 1,8 & 363,7 $\pm$ 1,9 & 521,6 $\pm$ 3,1  \\
\hline
MNIST & $\mathbf{0,814 \pm 0,010 }$& 0,808 $\pm$ 0,010 &  0,812 $\pm$ 0,008 & 0,772 $\pm$ 0,010 & 0,802 $\pm$ 0,009 & 0,800 $\pm$ 0,009 \\
\hline


\multicolumn{7}{|c|}{Сходимость алгоритмов, тыс. итераций  }  \\
%Выборка $\mathbf{X}$ & Алгоритм 1 & Алгоритм 2 & Алгоритм 3 & Алгоритм 4 & Алгоритм 5 & Алгоритм 6 \\
\hline
\specialcell{ Boston,  \\один  слой} &  25 & 25 & 25 & 14 & 10 & 27 \\
\hline
Boston, 3 слоя &  25 & 4 & 9 & 10 & 1 & 6 \\
\hline
Protein &   60 & 40 & 80 & 40 & 75 & 85 \\
\hline
MSD &  250 & 330 & 335 &  250 & 460 & 120  \\
\hline
MNIST &  1 & 6 & 3 &  13 & 3 & 25  \\
\hline
\end{tabular}
\end{table}





Модели имеют достаточно большое число параметров, поэтому в ходе оптимизации параметров может произойти переобучение. На выборке Boston Housing базовый алгоритм (1) показал наихудший результат в силу переобучения, при этом алгоритм 4 показал лучший результат по сравнению с алгоритмами 2 и 3. 
В данном случае использование вариационной оценки предпочтительнее алгоритмов, основанных на кросс-валидации. На выборке Protein все алгоритмы показали схожие результаты. На выборке MSD алгоритмы 4,5,6 показали худший результат в сравнении с алгоритмами, использующими валидационную подвыборку. Наихудший результат показал алгоритм 6, что говорит о значительном отличии апостериорного распределения параметров~\eqref{eq:posterior} от нормального.  

Алгоритм 6 показал низкое качество~\eqref{eq:rmse} при возмущении объектов выборки {в большинстве экспериментов}. В {трех} экспериментах наилучшие показатели по данному критерию показал алгоритм 4. Заметим, что алгоритм 5, являющийся модификацией алгоритма 4, показал худшие результаты как по RMSE, так и по RMSE при возмущении объектов выборки. 
{На выборке MNIST алгоритм 4 показал результаты значительно хуже остальных алгоритмов. В целом результаты по данному алгоритму схожи с результатами, описанными в~\cite{early}: в отличие от алгоритма 5 алгоритм 4, основанный на стохастическом градиентном спуске, дает заниженную оценку правдоподобия при приближении параметров к точке экстремума. } Алгоритм 5, основанный на динамике Ланжевена, также показал худшее время сходимости~{на выборках MSD и Protein}. Возможным дальнейшим улучшением качества этого алгоритма является введение дополнительной корректирующей матрицы, обеспечивающей лучшее время сходимости параметров к апостериорному распределению параметров~\cite{langevin}.

Программное обеспечение для проведения экспериментов и проверки результатов  находится в~\cite{my_src}. 












\chapter{Оптимизация гиперпараметров в задаче выбора модели}

Задача оптимизации гиперпараметров зависит как от критерия выбора модели, так и от метода оптимизации параметров модели.
Проиллюстрируем задачу оптимизации гиперпараметров \textit{двусвзяным байесовским выводом}. Для дальнейшей формализации задачи в общем виде введем переобозначение:
\begin{equation}
\label{eq:bayes0}
	\boldsymbol{\theta} = \mathbf{w}, \quad \mathbf{h} = [\alpha_1, \dots, \alpha_u],	
\end{equation}
где $\boldsymbol{\theta}$ --- множество оптимизируемых параметров модели, $\mathbf{h}$ --- множество гиперпараметров модели.

На \textit{первом уровне} байесовского вывода производится оптимизация параметров модели $f$ по заданной выборке $\mathfrak{D}$:
\begin{equation}
\label{eq:bayes1}
\hat{\boldsymbol{\theta}} = \argmax \bigl(-L(\boldsymbol{\theta}, \mathbf{h})\bigr) = p(\mathbf{w}|\mathbf{X}, \mathbf{y}, \mathbf{A}) = \frac{p(\mathbf{y}|\mathbf{X},\mathbf{w})p(\mathbf{w}|\mathbf{A})}{p(\mathbf{y}|\mathbf{X},\mathbf{A})}.
\end{equation}

На \textit{втором уровне} производится оптимизация апостериорного распределения гиперпараметров $\mathbf{h}$:
\[
p(\mathbf{A}|\mathbf{X}, \mathbf{y}) \propto p(\mathbf{y}|\mathbf{X},\mathbf{A})p(\mathbf{A}),
\]
где знак <<$\propto$>> означает равенство с точностью до нормирующего множителя.

Полагая распределение параметров $p(\mathbf{A})$ равномерным на некоторой большой окрестности, получим задачу оптимизации гиперпараметров:
\begin{equation}
\label{eq:bayes2}
	Q(\boldsymbol{\theta}, \mathbf{h}) = p(\mathbf{y}|\mathbf{X},\mathbf{A}) = \int_{\mathbf{w} \in \mathbb{R}^u} p(\mathbf{y}|\mathbf{X}, \mathbf{w}) p(\mathbf{w}|\mathbf{A}) \to \max_{[\alpha_1, \dots, \alpha_u] \in \mathbb{R}^{n}}.
\end{equation}


\begin{figure}
  \includegraphics[width=0.8\linewidth]{slide_plots/hyper.png}
\label{fig:hyper}
    \caption{Зависимость правдодобия модели от значения гиперпараметра $\alpha$. TODO: переделать}
 
   
    \end{figure}


Сфорумлируем задачу оптимизации гиперпараметров в общем виде. Обозначим за $\mathbf{h}  \in \mathbb{R}^h$ вектор гиперпараметров модели~\eqref{eq:bayes0}.   Обозначим за $\boldsymbol{\theta} \in \mathbb{R}^s$ множество всех оптимизируемых параметров~\eqref{eq:bayes0}. Пусть задана дифференцируемая функция потерь $L(\boldsymbol{\theta}, \mathbf{h})$, по которой производится оптимизация функции ${f}$~\eqref{eq:bayes1}. 
Пусть также задана дифференцируемая функция $Q(\boldsymbol{\theta}, \mathbf{h})$, определяющая итоговое качество модели ${f}$ и приближающая интеграл~\eqref{eq:bayes2}.

Требуется найти параметры $\hat{\boldsymbol{\theta}}$ и гиперпараметры $\hat{\mathbf{h}}$ модели, доставляющие минимум следующему функционалу:
\begin{equation}
\label{eq:main}
	\hat{\mathbf{h}} = \argmax_{\mathbf{h} \in \mathbb{R}^h} Q(\hat{\boldsymbol{\theta}}(\mathbf{h}), \mathbf{h}),
\end{equation}
\begin{equation}
\label{eq:main2}
	\hat{\boldsymbol{\theta}}(\mathbf{h}) =  \argmin_{\boldsymbol{\theta} \in \mathbb{R}^s} L(\boldsymbol{\theta}, \mathbf{h}).
\end{equation}

Рассмотрим вид переменной $\boldsymbol{\theta}$ и функций $L, Q$ для различных методов выбора модели и оптимизации ее параметров.

\textbf{Базовый метод}
Пусть оптимизация параметров и гиперпараметров производится по всей выборке $\mathfrak{D}$ по одной и той же функции:
$$L(\boldsymbol{\theta}, \mathbf{h}) = Q(\boldsymbol{\theta}) = \text{log}p(\mathbf{y}, \mathbf{w} | \mathbf{X}, \mathbf{A}) = \text{log} p(\mathbf{y}|\mathbf{X}, \mathbf{w})+\text{log}p(\mathbf{w}|\mathbf{A})$$.

Вспомогательная переменная $\boldsymbol{\theta}$, по которой производится оптимизация модели $f$,  соответствует параметрам модели: 
\[
\boldsymbol{\theta} = \mathbf{w}.
\]

\textbf{Кросс-валидация}
Разобьем выборку $\mathfrak{D}$ на $k$ равных частей:
\[
\mathfrak{D} = \mathfrak{D}_1 \sqcup \dots \sqcup \mathfrak{D}_k.
\]


Запустим $k$ оптимизаций модели, каждую на своей части выборки. Положим $\boldsymbol{\theta} = [\mathbf{w}_1, \dots, \mathbf{w}_k]$, где $\mathbf{w}_1, \dots, \mathbf{w}_k$ --- параметры модели при оптимизации $k$.
 
Положим функцию $L$ равной  среднему значению минус логарифма апостериорной вероятности по всем $k-1$ разбиениям $\mathfrak{D}$:
\begin{equation}
\label{eq:cv}
L(\boldsymbol{\theta}, \mathbf{h}) = -\frac{1}{k}\sum_{q=1}^k \bigl(\frac{k}{k-1}\text{log}p(\mathbf{y} \setminus \mathbf{y}_q|\mathbf{X}\setminus \mathbf{X}_q, \mathbf{w}_q) + \text{log}p(\mathbf{w}_q|\mathbf{A})\bigr).
\end{equation}

Положим функцию $Q$ равной среднему значению правдоподобия выборки по частям выборки $\mathfrak{D}_q$, на которых не проходила оптимизация параметров:
\[
Q(\boldsymbol{\theta}, \mathbf{h}) = \frac{1}{k}\sum_{q=1}^k k\text{log}p(\mathbf{y}_q|\mathbf{X}_q, \mathbf{w}_q).
\]

\textbf{Вариационная оценка правдоподобия}
Положим $L=-Q$, равной вариационной оценке правдоподобия модели:
\begin{equation} 
\label{eq:elbo}
\text{log}~p(\mathbf{y}|\mathbf{X},\mathbf{A})  
\geq 
-\text{D}_\text{KL} \bigl(q(\mathbf{w})||p(\mathbf{w}|\mathbf{A})\bigr) + \int_{\mathbf{w}} q(\mathbf{w})\text{log}~{p(\mathbf{y}|\mathbf{X},\mathbf{w},\mathbf{A})} d \mathbf{w}  \approx
\end{equation}
\[
\approx \sum_{i=1}^m \text{log}~p({y}_i|\mathbf{x}_i, \mathbf{w}_i) - D_\text{KL}\bigl(q (\mathbf{w}) || p (\mathbf{w}|\mathbf{A})\bigr) = -L(\boldsymbol{\theta}, \mathbf{h}) = Q(\boldsymbol{\theta}),
\]

где $q$ --- нормальное распределение с диагональной матрицей ковариаций:
\begin{equation}
\label{eq:diag}
	q \sim \mathcal{N}(\boldsymbol{\mu}_q, \mathbf{A}^{-1}_q),
\end{equation}
где $\mathbf{A}_q = \text{diag}[\alpha^q_1, \dots, \alpha^q_u]^{-1}$ --- диагональная матрица ковариаций, $\boldsymbol{\mu}_q$ --- вектор средних компонент.
Расстояние $D_\text{KL}$ между двумя гауссовыми величинами задается как 
\[
	D_\text{KL}\bigl(q (\mathbf{w}) || p (\mathbf{w}|\mathbf{f})\bigr) = \frac{1}{2} \bigl( \text{Tr} [\mathbf{A}\mathbf{A}^{-1}_q] + (\boldsymbol{\mu} - \boldsymbol{\mu}_q)^\mathsf{T}\mathbf{A}(\boldsymbol{\mu} - \boldsymbol{\mu}_q) - u +\text{ln}~|\mathbf{A}^{-1}| - \text{ln}~|\mathbf{A}_q^{-1}| \bigr).
\]

В качестве оптимизируемых параметров $\boldsymbol{\theta}$ выступают параметры распределения $q$:
\[
\boldsymbol{\theta} = [\alpha_1, \dots, \alpha_u, {\mu}_1,\dots,{\mu}_u].
\]




\section{Градиентные методы оптимизации гиперпараметров}
Рассмотрим случай, когда оптимизация~\eqref{eq:main2} параметров $\boldsymbol{\theta}$ производится с использованием градиентных методов. 

\textbf{Определение.} Назовем оператором оптимизации алгоритм $T$ выбора вектора параметров $\boldsymbol{\theta}'$  по параметрам предыдущего шага $\boldsymbol{\theta}$:
\[
	\boldsymbol{\theta}' = T(\boldsymbol{\theta}, \mathbf{h}).
\]

Рассмотрим оператор градиентного спуска, производящий $\eta$ шагов оптимизации:
\begin{equation}
\label{eq:gd}
	 \hat{\boldsymbol{\theta}} = T \circ T \circ \dots \circ T(\boldsymbol{\theta}_0, \mathbf{h}) = T^\eta(\boldsymbol{\theta}_0, \mathbf{h}),
\end{equation}
где 
$$
	T(\boldsymbol{\theta}, \mathbf{h}) =\boldsymbol{\theta} - \gamma \nabla L(\boldsymbol{\theta}, \mathbf{h}), 
$$
$\gamma$ --- длина шага градиентного спуска, $\boldsymbol{\theta}_0$ --- начальное значение параметров $\boldsymbol{\theta}$. В данной работе в качестве опреатора оптимизации параметров модели выступает стохастический градиентный спуск:
\[
T(\boldsymbol{\theta}, \mathbf{h})_\text{SGD} =\boldsymbol{\theta} - \gamma \nabla L(\boldsymbol{\theta}, \mathbf{h})|_{\mathfrak{D} = \hat{\mathfrak{D}}},
\]
где $\hat{\mathfrak{D}}$ --- случайная подвыборка исходной выборки $\mathfrak{D}$.

Перепишем задачу оптимизации~\eqref{eq:main}, ~\eqref{eq:main2} в следующем виде
\begin{equation}
\label{eq:optim}
	\hat{\mathbf{h}} = \argmax_{\mathbf{h} \in \mathbb{R}^h} Q( T^\eta(\boldsymbol{\theta}_0, \mathbf{h})),
\end{equation}
где $\boldsymbol{\theta}_0$ --- начальное значение параметров $\boldsymbol{\theta}$.

Оптимизационную задачу~\eqref{eq:optim} предлагается решать с использованием градиентного спуска. Вычисление градиента от функции $Q( T^\eta(\boldsymbol{\theta}_0, \mathbf{h}))$ по гиперпараметрам $\mathbf{h}$ является вычислительно сложным в силу внутренней процедуры оптимизации $T(\boldsymbol{\theta}_0, \mathbf{h})$. 
Общая схема  оптимизации гиперпараметров представлена следующим образом:
\begin{enumerate}
\item От 1 до  $l$:
\item Инициализировать парметры $\boldsymbol{\theta}$ при условии гиперпараметров $\mathbf{h}$.
\item Приближенно решить задачу оптимизации~\eqref{eq:optim} и получить новый вектор параметров $\mathbf{h}'$
\item $\mathbf{h} = \mathbf{h}'$.
\end{enumerate}
где $l$ --- количество итераций оптимизации гиперпараметров. Рассмотрим методы приближенного решения данной задачи оптимизации.



\textbf{Жадный алгоритм}
В качестве  правила обновления вектора гиперпараметров $\mathbf{h}$ на каждом шаге оптимизации~\eqref{eq:gd} выступает градиентный спуск с учетом обновления параметров $\boldsymbol{\theta}$ на данном шаге:
\[
	\mathbf{h}' = \mathbf{h} - \gamma_{\mathbf{h}} \nabla_{\mathbf{h}}  Q \bigl(T(\boldsymbol{\theta}, \mathbf{h}) , \mathbf{h}\bigr) = \mathbf{h} - \gamma_{\mathbf{h}} \nabla_{\mathbf{h}}  Q\bigl(\boldsymbol{\theta} - \gamma \nabla L(\boldsymbol{\theta}, \mathbf{h}), \mathbf{h})\bigr),
\]
где $\gamma_{\mathbf{h}}$ --- длина шага оптимизации гиперпараметров.

\textbf{Алгоритм HOAG}
Предлагается получить приближенное значения градиента гиперпараметров $\nabla_{\mathbf{h}} Q \bigl(T^\eta(\boldsymbol{\theta}_0, \mathbf{h})\bigr)$ на основе следующей формулы:
\[
\nabla_{\mathbf{h}} Q \bigl(T^\eta(\boldsymbol{\theta}_0, \mathbf{h})\bigr) = \nabla_{\mathbf{h}} Q(\boldsymbol{\theta}, \mathbf{h}) - (\nabla^2_{\boldsymbol{\theta}, \mathbf{h}} L(\boldsymbol{\theta}, \mathbf{h}))^\text{T}\mathbf{H}(\boldsymbol{\theta})^{-1}\nabla_{\boldsymbol{\theta}} Q(\boldsymbol{\theta}, \mathbf{h}),
\]
где $\mathbf{H}$ --- гессиан функции $L$ по параметрам $\boldsymbol{\theta}$.

Процедура получения приближенного значения градиента гиперпараметров $\nabla_{\mathbf{h}} Q$  производится итеративно:
\begin{enumerate}
\item Провести $\eta$ шагов оптимизации: $\boldsymbol{\theta} = T(\boldsymbol{\theta}_0, \mathbf{h})$.
\item Решить линейную систему для вектора $\boldsymbol{\lambda}$: $\mathbf{H}(\boldsymbol{\theta})\boldsymbol{\lambda} =  \nabla_{\boldsymbol{\theta}} Q(\boldsymbol{\theta}, \mathbf{h})$.
\item Приближенное значение градиентов гиперпараметра вычисляется как: $\hat{\nabla}_{\mathbf{h}}Q = \nabla_{\mathbf{h}}Q(\boldsymbol{\theta}, \mathbf{h}) -\nabla_{\boldsymbol{\theta}, \mathbf{h}} L(\boldsymbol{\theta}, \mathbf{h})^T\boldsymbol{\lambda}$.
\end{enumerate}

Итоговое правило обновления:
\begin{equation}
\label{eq:update_hyper}
\mathbf{h}' = \mathbf{h} - \gamma_{\mathbf{h}} \hat{\nabla}_{\mathbf{h}}Q.
\end{equation}

В данной работе для приближенного решения  шага 2 алгоритма HOAG используется стохастический градиентный спуск в силу сложности вычисления гессиана $\mathbf{H}(\boldsymbol{\theta})$.


\textbf{Алгоритм DrMad}

Для получения градиента от оптимизируемой функции $Q$ как от функции от начальных параметров $\boldsymbol{\theta}_0$ предлагаетя пошагово восстановить $\eta$ шагов оптимизации $T(\boldsymbol{\theta}_0)$ в обратном порядке аналогично методу обратного распространения ошибок. Для упрощения данной процедуры вводится предположение,что траектория изменения параметров $\boldsymbol{\theta}$ линейна:
\begin{equation}
\label{eq:mad_lin}
\boldsymbol{\theta}^\tau = \boldsymbol{\theta}_0 + \frac{\tau}{\eta} T(\boldsymbol{\theta}).
\end{equation}

Алгоритм вычисления приближенного значения градиента $\nabla \mathbf{h}$ является частным случаем алгоритма обратного распространения ошибки и представим в следующем виде:
\begin{enumerate}
\item Провести $\eta$ шагов оптимизации: $\boldsymbol{\theta} = T(\boldsymbol{\theta}_0, \mathbf{h})$.
\item Положим $\hat{\nabla} \mathbf{h} = \nabla_\mathbf{h} Q(\boldsymbol{\theta}, \mathbf{h}).$ 
\item Положим $d\mathbf{v} = \mathbf{0}.$
\item Для $\tau = \eta \dots 1 $ повторить:
\item Вычислить значения параметров $\boldsymbol{\theta}^\tau$\eqref{eq:mad_lin}.
\item $d\mathbf{v} =  \gamma \hat{\nabla}_{\boldsymbol{\theta}}$.
\item $\hat{\nabla} \mathbf{h} =  \hat{\nabla} \mathbf{h} - d\mathbf{v}\nabla_{\mathbf{h}} \nabla_{\boldsymbol{\theta}} Q$.
\item $\hat{\nabla} \boldsymbol{\theta}  = \hat{\nabla} \boldsymbol{\theta}  - d\mathbf{v}\nabla_{\boldsymbol{\theta}} \nabla_{\boldsymbol{\theta}} Q$.
\end{enumerate}

Итоговое правило обновления гиперпараметров аналогично~\eqref{eq:update_hyper}.
В работе~\cite{hyper_mad} отмечается неустойчивость алгоритма при высоких значениях длины шага градиентного спуска $\gamma$. Поэтому вместо исходного правила~\eqref{eq:mad_lin} в данной работе первые 5\% значений параметров не рассматриваются, а также учитывается только каждый $\tau_k$ шаг оптимизации:
\begin{equation}
\label{eq:mad_lin2}
\boldsymbol{\theta}^\tau = \boldsymbol{\theta}_{\tau_0} + \frac{\tau}{\eta} T(\boldsymbol{\theta}), \quad \tau \in \{\tau_0,\dots,\eta\}, \tau \text{ mod } \tau_k = 0,
\end{equation}
где $\tau_0 = [0.05 \cdot \eta]$.


\chapter{Анализ прикладных задач порождения и выбора моделей глубокого обучения}
\addcontentsline{toc}{section}{Заключение}
\chapter*{Заключение}


\clearpage 
\addcontentsline{toc}{section}{Список иллюстраций}
\listoffigures

\clearpage
\addcontentsline{toc}{section}{Список таблиц}
\listoftables

\clearpage
\addcontentsline{toc}{section}{Список литературы}
\renewcommand{\bibname}{Список использованных источников}
\addcontentsline{toc}{chapter}{Список использованных источников}
\bibliographystyle{gost71u}
\bibliography{dis_literature}


\end{document}
