\noindent$\mathbf{x}_i$ --- вектор признакового описания $i$-го объекта\\
$y_i$ --- метка $i$-го объекта\\
$\mathfrak{D}$ --- выборка\\
$\mathbf{X}$ --- матрица, содержащая признаковое описание объектов выборки\\
$\mathbf{y}$ --- вектор меток объектов выборки\\
$m$ --- количество объектов в выборке\\
$n$ --- количество признаков в признаковом описании объекта\\
$\mathbb{X}$ --- признаковое пространство объектов\\
$\mathbb{Y}$ --- множество меток объектов\\
$R$ --- множество классов в задаче классификации\\
$(V,E)$ --- граф со множеством вершин $V$ и множеством ребер $E$\\
$\mathbf{g}^{j,k}$ --- вектор базовых функций для ребра $(j,k)$\\
$K^{j,k}$ --- мощность вектора базовых функций для ребра $(j,k)$\\
$\textbf{agg}_v$ --- функция аггрегации для вершины $v$. 
$\boldsymbol{\gamma}^{j,k}$ --- структурный параметр для ребра $(j,k)$\\
$\Delta^{K}$ --- симплекс на $K$ вершинах\\
$\bar{\Delta}^{K}$ --- множество вершин симплекса на $K$ вершинах\\
$\mathfrak{F}$ --- параметрическое семейство моделей\\
$\mathbf{w}$ --- параметры модели\\
$\mathbb{W}$ --- пространство параметров модели\\
$\boldsymbol{\Gamma}$ --- структура модели\\
$\amsmathbb{\Gamma}$ --- множество значений структуры модели\\
$\mathbf{h}$ --- гиперпараметры модели\\
$\mathbb{H}$ --- пространство гиперпараметров модели\\
$p(\mathbf{w}, \boldsymbol{\Gamma}|\mathbf{h})$ --- априорное распределение параметров и структуры модели\\
$p(\mathbf{w}, \boldsymbol{\Gamma}|\mathbf{y}, \mathbf{X}, \mathbf{h})$ --- апостериорное распределение параметров и структуры модели\\
$p({y}, \mathbf{w},  \boldsymbol{\Gamma}|\mathbf{x}, \mathbf{h})$ --- вероятностная модель глубокого обучения\\
$p(y|\mathbf{X}, \mathbf{w}, \boldsymbol{\Gamma})$ --- правдоподобие выборки\\
$p(y|\mathbf{X}, \mathbf{h})$ --- правдоподобие модели\\
$q(\mathbf{w}, \boldsymbol{\Gamma})$ --- вариационное распределение\\
$\boldsymbol{\theta} \in \mathbb{R}^u$ --- вариационные параметры\\
$L(\boldsymbol{\theta}, \mathbf{h}, \mathbf{X}, \mathbf{y})$ --- функция потерь\\
$Q(\mathbf{h}, \boldsymbol{\theta},  \mathbf{X}, \mathbf{y})$ --- валидационная функция\\
$T(\boldsymbol{\theta}| L, \mathbf{y}, \mathbf{X},  \mathbf{h}, \boldsymbol{\beta})$ --- оператор оптимизации\\
$\boldsymbol{\lambda}$ --- вектор метапараметров\\
$\amsmathbb{\Lambda}$ --- пространство метапараметров\\
$\mathfrak{Q}$ --- семейство вариационные распределений\\
$\mathsf{S}$ --- энтропия распределения\\
$M$ --- множество моделей без общей параметризации\\
