\noindent$\x_i \in \X$ --- вектор признакового описания $i$-го объекта\\
$y_i \in \y$ --- метка $i$-го объекта\\
$\D$ --- выборка\\
$\X \subset \Xb$ --- матрица, содержащая признаковое описание объектов выборки\\
$\y \subset \yb$ --- вектор меток объектов выборки\\
$m$ --- количество объектов в выборке\\
$n$ --- количество признаков в признаковом описании объекта\\
$\Xb = \mathbb{R}^m$ --- признаковое пространство объектов\\
$\yb$ --- множество меток объектов\\
$R$ --- множество классов в задаче классификации\\
$r$ --- число оптимизаций модели\\
$(V,E)$ --- граф со множеством вершин $V$ и множеством ребер $E$\\
$\mathbf{g}^{j,k}$ --- вектор базовых функций для ребра $(j,k)$\\
$K^{j,k}$ --- мощность вектора базовых функций для ребра $(j,k)$\\
$\textbf{agg}_v$ --- функция агрегации для вершины $v$ \\
$\boldsymbol{\gamma}^{j,k}$ --- структурный параметр для ребра $(j,k)$\\
$\Delta^{K}$ --- симплекс на $K$ вершинах\\
$\bar{\Delta}^{K}$ --- множество вершин симплекса на $K$ вершинах\\
$\F$ --- параметрическое семейство моделей\\
$U$ --- область определения оптимизационной задачи\\
$\w \in \Wb$ --- параметры модели\\
$\Wb$ --- пространство параметров модели\\
$\Uw \subset \Wb$ --- область определения параметров модели\\
$\G \in \Gb$ --- структура модели\\
$\Gb$ --- множество значений структуры модели\\
$\UG \subset \Gb$ --- область определения параметров модели\\
$\h \in \Hb$ --- гиперпараметры модели\\
$\Hb$ --- пространство гиперпараметров модели\\
$\Uh \subset \Hb$ --- область определения гиперпараметров\\
$\teta \in \Tetab$ --- параметры вариационного распределения\\
$\Tetab$ --- пространство параметров вариационного распределения\\
$\Uteta \subset \Tetab$ --- область определения вариационных параметров модели\\
$\tetaw \in \Tetawb$ --- параметры вариационного распределения, аппроксимирующего апостериорное распределение параметров модели\\
$\Tetawb$ --- пространство параметров вариационного распределения, аппроксимирующего апостериорное распределение параметров модели\\
$\Utetaw \subset \Tetawb$ --- область определения параметров вариационного распределения, аппроксимирующего апостериорное распределение параметров модели\\
$\tetaG \in \TetaGb$ --- параметры вариационного распределения, аппроксимирующего апостериорное распределение структуры модели\\
$\TetaGb$ ---пространство параметров вариационного распределения, аппроксимирующего апостериорное распределение структуры модели\\
$\UtetaG \subset \TetaGb$ --- область определения параметров вариационного распределения, аппроксимирующего апостериорное распределение структуры модели\\
$\lam \in \Lamb$ --- вектор метапараметров\\
$\Lamb$ --- пространство метапараметров\\
$\Ulam \subset \Lamb$ --- область определения  метапараметров\\
$\LL$ --- правдоподобие выборки\\
$\prior$ --- априорное распределение параметров и структуры модели\\
$\priorh$ ---  распределение гиперпараметров модели\\
$\priorG$ --- априорное распределение структуры модели\\
$\priorw$ --- априорное распределение  параметров модели\\
$\post$ --- апостериорное распределение параметров и структуры модели\\
$\postw$ --- апостериорное распределение структуры модели\\
$\postG$ --- апостериорное распределение  структуры модели\\
$\posth$ --- апостериорное распределение  гиперпараметров \\
$p({y}, \mathbf{w},  \boldsymbol{\Gamma}|\mathbf{x}, \mathbf{h})$ --- вероятностная модель глубокого обучения\\
$\EV$ --- обоснованность модели\\
$\q$ --- вариационное распределение параметров и структуры модели\\
$\qw$ --- вариационное распределение структуры модели\\
$\qG$ --- вариационное распределение параметров модели\\
$\Loss$ --- функция потерь\\
$\Val$ --- валидационная функция\\
$T(\boldsymbol{\theta} |L(\boldsymbol{\theta} |\mathbf{y},\mathbf{X},\mathbf{h},\boldsymbol{\lambda}))$ --- оператор оптимизации\\
$\mathfrak{Q}$ --- семейство вариационных распределений\\
$\mathsf{S}$ --- энтропия распределения\\
$M$ --- множество моделей без общей параметризации\\
$\KL{p_1}{p_2}$ --- дивергенция Кульбака-Лейблера  между распределениями $p_1$ и $p_2$
$\A^{-1}$ --- матрица ковариаций параметров модели\\
$\s$ --- конкатенация параметров концентрации на структуре модели
